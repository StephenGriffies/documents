%LATEX 
%\documentclass[10pt,draft]{beamer}
\documentclass[10pt]{beamer}
\usepackage{beamerthemesplit} % new 


\usepackage{bm,amsmath,amssymb}
\usepackage{mathtools,slashed,esint}

\usepackage{animate}


\usepackage{pgf}
\usepackage{wasysym}
\usepackage{xcolor}
\usepackage[english]{babel}
\usepackage[latin1]{inputenc}
\usepackage{times}
\usepackage[T1]{fontenc}
\usepackage{wasysym}
\usepackage{centernot}

\usepackage{tikz}
\usepackage{pgf}
\usepackage{xcolor}
\usepackage[english]{babel}
\usepackage[latin1]{inputenc}
\usepackage{times}
\usepackage[T1]{fontenc}

%\usepackage{graphics}
\usepackage{graphicx}
\usepackage{media9}
\usepackage{natbib}
\usepackage{color}

%\usepackage{../../../../../Documents/TEX/styles/mystyle_beamer}

\usepackage{ragged2e}
\usepackage{etoolbox}
\usepackage{lipsum}

\apptocmd{\frame}{\justifying}{}{}

\usepackage{hyperref}  % must be used last in the list of packages 

\hypersetup{colorlinks=true,linkcolor=green,citecolor=green,filecolor=green,urlcolor=green} 


\setlength\fboxsep{0pt}
\setlength\fboxrule{0.5pt}

\newcommand\myeq{\mathrel{\overset{\makebox[0pt]{\mbox{\normalfont\tiny\sffamily M constant}}}{=}}}
\newcommand\myeqq{\mathrel{\overset{\makebox[0pt]{\mbox{\normalfont\tiny\sffamily Stokes Thm}}}{=}}}

\newcommand{\notimplies}{%
  \mathrel{{\ooalign{\hidewidth$\not\phantom{=}$\hidewidth\cr$\implies$}}}}

\newcommand{\Simley}[1]{%
\begin{tikzpicture}[scale=0.11]
    \newcommand*{\SmileyRadius}{1.0}%
    \draw [fill=brown!10] (0,0) circle (\SmileyRadius)% outside circle
        %node [yshift=-0.22*\SmileyRadius cm] {\tiny #1}% uncomment this to see the smile factor
        ;  

    \pgfmathsetmacro{\eyeX}{0.5*\SmileyRadius*cos(30)}
    \pgfmathsetmacro{\eyeY}{0.5*\SmileyRadius*sin(30)}
    \draw [fill=cyan,draw=none] (\eyeX,\eyeY) circle (0.15cm);
    \draw [fill=cyan,draw=none] (-\eyeX,\eyeY) circle (0.15cm);

    \pgfmathsetmacro{\xScale}{2*\eyeX/180}
    \pgfmathsetmacro{\yScale}{1.0*\eyeY}
    \draw[color=red, domain=-\eyeX:\eyeX]   
        plot ({\x},{
            -0.1+#1*0.15 % shift the smiley as smile decreases
            -#1*1.75*\yScale*(sin((\x+\eyeX)/\xScale))-\eyeY});
\end{tikzpicture}%
}%



% beamer prologue begins here
\mode<presentation>
{                               % see user guide for choices
  \usetheme{Frankfurt}
%  \usetheme{Madrid}
%  \usecolortheme{seahorse}
%  \usecolortheme[RGB={205,173,0}]{structure} 
  \usecolortheme[RGB={0,100,180}]{structure} 
  \usefonttheme[onlymath]{serif}
  \setbeamercovered{transparent} % or whatever (possibly just delete it)
}

% drop those navigation symbols at bottom right
%\setbeamertemplate{navigation symbols}{} 
\setbeamertemplate{navigation symbols}{}
%\setbeamertemplate{footline}[frame number]



\title[Ocean circulation]{Ocean circulation
  as a problem in mathematical \& computational physics:  a historical
  and contemporary  perspective} % (optional)

\subtitle{ {\it AMSI Summer School in the Mathematical Sciences} \\
  30 Jan 2019 \\ \vspace{.25cm}
  University of New South Wales \\
 Sydney, Australia  \\ Traditional lands of the Bedegal people of the Eora Nation
%\vspace{.25cm}
%\fbox{\includegraphics[width=.3\textwidth]{../../../../../Griffies/lectures/India_2013/griffies/figs/cm2p6.png}}
}

\author[{\sc Stephen.M.Griffies@gmail.com}] % short form for footer, use if many authors
{ {\sc Stephen.Griffies@noaa.gov  \\ SMG@Princeton.edu \\  Stephen.M.Griffies@gmail.com 
} 
}
% - Use the \inst{?} command only if the authors have different 
%   affiliation. 




\institute[NOA/GFDL + Princeton University]{NOAA/GFDL + Princeton University} 

%\date{\today}
\date{}


%\logo{\includegraphics[height=5mm]{../../../../../Documents/images/PrincetonLogo.png}} 
\logo{\includegraphics[height=5mm]{../../../../../Documents/images/noaa-logo.png}} 

\subject{Talks} % for "PDF information catalog", can be left out. 

% this shows the outline at the beginning of every section,
% highlighting the current section
 \AtBeginSection[]
 {
   \begin{frame}
     \frametitle{Outline}
 \tableofcontents[ 
 currentsubsection, 
 hideothersubsections, 
 sectionstyle=show/hide, 
 subsectionstyle=show/hide, 
 ] 
   \end{frame}
 }


\setlength{\unitlength}{1mm}  % used in picture environment

\begin{document}

  
% beamer makes the titlepage from info above: author, date, title, etc.
\begin{frame}
  \titlepage
\end{frame}


\begin{frame}
  \frametitle{Goals, Assumptions, Apologies}


\begin{exampleblock}{}
\begin{itemize}

\item This talk offers a taste for the mathematics and physics
  encountered when studying ocean circulation:
  \begin{itemize}
    \item[$\star$] A rapid-fire historical introduction to geophysical fluid mechanics.  
    \item[$\star$] Apologies for those who took Keating and
      Alexander's course ``Mathematics of Planet Earth'', as there
      will be some overlap.
  \end{itemize}

\item With Australia's finest math/science students in the audience, I
  assume you wish to see a few of the details even for this public lecture. 
  \begin{itemize}
  \item[$\star$] Theoretical physical oceanography is a rich
    sub-discipline of mathematical \& computational physics with
    sophisticated methods fostering understanding and improving
    predictions.
  \end{itemize}

\item We will encounter animations from state-of-the-science computational
  fluid mechanics simulations of ocean flows.
  \begin{itemize}
  \item[$\star$] Eye-candy to illustrate the richness, complexity, and
    beauty of ocean flows.  
  \item[$\star$] Animations also offer us a short rest.
  \end{itemize}

\end{itemize}
\end{exampleblock}{}

\end{frame}



\begin{frame}
  \frametitle{Some maths/physics relevant to ocean circulation}

%  We make use of a range of mathematical and physical tools/ideas for
%  studying ocean fluid mechanics, including the following: 

\vspace{-.5cm} 

\begin{exampleblock}{}
\begin{itemize}

\small 

\item Linear \& nonlinear partial differential equations of continuum
  mechanics;

\item Geometrical/tensor methods to pose the governing equations on a
  rotating and stratified sphere;

\item Numerical methods for computer models of use as a tool for
  experimental investigations;

\item Geometrical/tensor methods to analyze the space-time properties
  of complex and multi-scale flows emerging from the governing
  equations;

\item Dynamical systems and stochastic methods to conceptually frame
  multi-scale linear, nonlinear, and chaotic behaviour of currents,
  waves, instabilities, coherent structures, and turbulence.

\item Statistical methods for robust analysis and probabilistic
  methods for prediction;

\item Big data science tools to reveal underlying patterns to help
  diagnose physical processes from within the huge data streams coming
  from satellites and simulations.

\normalsize 

\end{itemize}

\end{exampleblock}{}

\end{frame}


\begin{frame}
  \frametitle{Animation: sea surface temperature and currents}

\begin{center}
\vspace{-.25cm}
\fbox{\includegraphics[width=0.7\textwidth]{./images/pictures/gaea.jpg}}
\end{center}

\begin{center}
Computer model simulation of sea surface temperature from the
\href{https://vimeo.com/27076776}{NOAA/GFDL/Princeton climate model
  CM2.6} run on the Gaea supercomputer of the US Department of Energy. 
\end{center}

\end{frame}



% beamer makes the ToC from sections and subsections below...
 \begin{frame}
   \frametitle{Outline}
   \tableofcontents
 \end{frame}


\section{Building the math/physics framework}


\begin{frame}
  \frametitle{Archimedes of Syracuse: buoyancy}

\begin{center}
\vspace{-.25cm}
\fbox{\includegraphics[width=0.22\textwidth]{./images/people_pre_16thcentury/Domenico-Fetti_Archimedes_1620.jpg}}
\hspace{2cm}  
\fbox{\includegraphics[width=0.38\textwidth]{./images/climate/WunschFerrari_fig1.pdf}}
\center{\scriptsize \href{https://en.wikipedia.org/wiki/Archimedes}{Archimedes (287 - c.212 BC)}
\hspace{1.5cm}  Pacific buoyancy
(\href{}{Wunsch \& Ferrari 2004)} }
\end{center}

\begin{exampleblock}{}
\begin{itemize}
\item ``Eureka Moment'': an understanding of how the buoyancy force on an object in a
  liquid equals to the weight of displaced liquid.
\begin{equation}
 {\bm F}^{\mbox{\tiny buoyancy}} = \hat{\bm z} \, g \, \rho_{\mbox{\tiny liquid}} \, V_{\mbox{\tiny displace}}.
\end{equation}

\item Gravity stratifies according to buoyancy: light water above heavy.

\item Buoyancy is built into the hydrostatic balance:
  $\mathrm{d} p/\mathrm{d} z = -\rho \, g$: pressure at a point equals
  to the weight per area above the point.

\end{itemize}
\end{exampleblock}{}

\end{frame}


\begin{frame}
  \frametitle{Leonardo  di ser Piero da Vinci: visualizing fluid flow}

\begin{center}
\vspace{-.25cm}
\fbox{\includegraphics[width=0.31\textwidth]{./images/people_pre_16thcentury/Francesco_Melzi_-_Portrait_of_Leonardo_-_WGA14795.jpg}}
\hspace{1.5cm}  
\fbox{\includegraphics[width=0.31\textwidth]{./images/people_pre_16thcentury/Studies_of_Water_passing_Obstacles_and_falling.jpg}}
\center{\scriptsize \href{https://en.wikipedia.org/wiki/Leonardo_da_Vinci}{Leonardo da Vinci (1452-1519)}
\hspace{1cm}  Flow around obstacles \& entering a pool}
\end{center}

\begin{exampleblock}{}
\begin{itemize}
\item Leonardo's observations and visualizations of laminar and
  turbulent fluid flows were both insightful and prescient.
\item Detail in his sketches suggest an understooding of vorticity \&
  vortex tubes: ${\bm \omega} = \nabla \wedge {\bm v}$.
\end{itemize}
\end{exampleblock}{}

\end{frame}



\begin{frame}
  \frametitle{Newton \& LaPlace: mechanics \& math physics}

\begin{center}
\vspace{-.25cm}
\fbox{\includegraphics[width=0.23\textwidth]{./images/people_17-19th_century/GodfreyKneller-IsaacNewton-1689.jpg}}
\hspace{2cm} 
\fbox{\includegraphics[width=0.24\textwidth]{./images/people_17-19th_century/Laplace.jpg}}
\vspace{-.3cm}
\center{\scriptsize \href{https://en.wikipedia.org/wiki/Isaac_Newton}{Newton (1642-1727)}
\hspace{2cm}  \href{https://en.wikipedia.org/wiki/Pierre-Simon_Laplace}{LaPlace (1749-1827)}}
\end{center}

\vspace{-.2cm}
\begin{exampleblock}{}
\begin{itemize}
\item Foundations for mechanics established by Newton.
\item Newton's 2nd Law of Motion: fundamental to all classical mechanical systems, including fluids 
\begin{equation}
   {\bm F} = \frac{\mathrm{d}{\bm P}}{\mathrm{d}t}  \; \; \; \;  \myeq \; \; \; \;   M \, \frac{\mathrm{d}^{2} {\bm X}}{\mathrm{d}t^{2}} = M \, {\bm A}
  \qquad \left( {\bm P} = \frac{\mathrm{d}{\bm X}}{\mathrm{d}t}   \right).
\end{equation}
\item LaPlace used Newtonian mechanics (with Coriolis force) to
  describe ocean tides: {\it LaPlace's tidal equations}.
\end{itemize}
\end{exampleblock}{}


\end{frame}


\begin{frame}
  \frametitle{Coriolis: motion in a rotating reference frame}

\begin{center}
\vspace{-.25cm}
\fbox{\includegraphics[width=0.18\textwidth]{./images/people_17-19th_century/Gaspard-Gustave_de_Coriolis.jpg}}
\hspace{1cm}
{\includegraphics[width=.6\textwidth]{./images/schematics/geostrophy_forces.pdf}}
\vspace{-.2cm}
\center{\scriptsize \href{https://en.wikipedia.org/wiki/Gaspard-Gustave_de_Coriolis}{Coriolis (1792-1843)}
  \hspace{1cm}  Geostrophic balance around a northern hemisphere low}

\end{center}


\begin{exampleblock}{}
\begin{itemize}
\item Motion in a rotating reference frame introduces a non-inertial
  force that deflects motion to the right in northern hemisphere and
  left in the southern hemisphere 
\begin{equation}
   {\bm F}_{\mbox{\tiny Coriolis}}  = - 2 \, \rho \,  {\bm \Omega} \wedge  {\bm v}.
\end{equation}

\item Allows for steady {\it geostrophic balance} between pressure and Coriolis forces 
\begin{equation}
 {\bm F}_{\mbox{\tiny Coriolis}}  +  {\bm F}_{\mbox{\tiny pressure}}  = 0  \Longrightarrow 
   2 \, \rho \,  {\bm \Omega} \wedge  {\bm v} = -\nabla p. 
\end{equation}

\end{itemize}
\end{exampleblock}{}



\end{frame}





\begin{frame}
  \frametitle{Fluid dynamical equations for ocean motion}

\begin{center}
\vspace{-.25cm}
\fbox{\includegraphics[width=0.15\textwidth]{./images/people_17-19th_century/Claude-Louis_Navier.jpg}}
\hspace{1.5cm} 
\fbox{\includegraphics[width=0.15\textwidth]{./images/people_17-19th_century/Ggstokes.jpg}}

\center{\scriptsize 
\href{https://en.wikipedia.org/wiki/Claude-Louis_Navier}{Navier (1785-1836)}
\hspace{1.5cm}
\href{https://en.wikipedia.org/wiki/Sir_George_Stokes,_1st_Baronet}{Stokes (1819-1903)}
}

\end{center}

\begin{exampleblock}{}
\begin{itemize}
\item Newton and LaPlace's calculus and mechanics is the foundation to
  the dynamical equation for a continuous fluid media.

\item Adding Coriolis force yields the ocean dynamical equations
\begin{equation}
  \underbrace{ \frac{\partial {\bm v}}{\partial t}}_{\mbox{\tiny local acceleration}}
  + 
  \underbrace{ ({\bm v} \cdot \nabla ) \, {\bm v} }_{\mbox{\tiny advection by flow}} 
 +
  \underbrace{ 2 \, {\bm \Omega} \wedge {\bm v} }_{\mbox{\tiny Coriolis}} 
= \underbrace{ -\frac{1}{\rho} \nabla p}_{\mbox{\tiny pressure}}
   + 
  \underbrace{ \rho \, g \, \hat{\bm z}}_{\mbox{\tiny gravity}}
 + 
  \underbrace{ \nabla \cdot {\bm \tau}_{\mbox{\tiny stress}}}_{\mbox{\tiny friction}}
\end{equation}

\item This equation is relevant for nearly all observed fluid motions
  from galaxies to blood.

\end{itemize}
\end{exampleblock}{}

\end{frame}



\begin{frame}
  \frametitle{Euler and Lagrange: dual views of fluid motion}

\begin{center}
\vspace{-.3cm}
\fbox{\includegraphics[width=0.15\textwidth]{./images/people_17-19th_century/Leonhard_Euler.jpg}}
\hspace{2cm} 
\fbox{\includegraphics[width=0.17\textwidth]{./images/people_17-19th_century/Lagrange_portrait.jpg}}

\vspace{-.25cm}
\center{\scriptsize 
\href{https://en.wikipedia.org/wiki/Leonhard_Euler}{Euler (1707 - 1783)}
\hspace{1.5cm}
\href{https://en.wikipedia.org/wiki/Joseph-Louis_Lagrange}{Lagrange (1736-1813)}
}
\end{center}


\begin{exampleblock}{}
\begin{itemize}

\item Eulerian frame: fluid viewed/measured in a frame fixed in space ({\it laboratory frame}).  
\item Lagrangian frame: fluid viewed/measured in a frame fixed on a fluid element ({\it material frame}).

\end{itemize}
\end{exampleblock}{}


\begin{center}
\vspace{-.25cm}
\fbox{\includegraphics[width=0.55\textwidth]{./images/schematics/Euler_and_Lagrange_viewpoints.pdf}}
\end{center}


\end{frame}



\begin{frame}
  \frametitle{Transport by waves and eddies: Stokes Drift}

\begin{center}
\vspace{-.25cm}
{\includegraphics[width=\textwidth]{./images/schematics/StokesDrift.pdf}}
\vspace{-.5cm}
\center{\scriptsize Surface Stokes drift \hspace{2cm} Layer  thickness
  transport \hspace{1.6cm} APE release}
\end{center}

\begin{exampleblock}{}
\begin{itemize}
\item Stokes identified how waves can move matter: {\it Stokes drift}.
    \begin{itemize}
         \item[$\star$] Think of motion just outside the breaking surf-zone.  
    \end{itemize}

  \item Stokes drift leads to the ``bolus'' transport within buoyancy
    layers.
    \begin{itemize}
    \item[$\star$] Think of how a snake swallows its meal.
    \end{itemize}

  \item A generalized version of Stokes drift arises as sloping
    buoyancy surfaces flatten and release potential energy during
    baroclinic instability.


\end{itemize}
\end{exampleblock}{}

\end{frame}


\begin{frame}
  \frametitle{Kelvin and Helmholz: circulation and vorticity}

\begin{center}
\vspace{-.3cm}
\fbox{\includegraphics[width=0.23\textwidth]{./images/people_17-19th_century/Lord_Kelvin_photograph.jpg}}
\hspace{1cm} 
{\includegraphics[width=0.25\textwidth]{./images/schematics/vorticity_and_circulation.pdf}}
\hspace{1cm}
\fbox{\includegraphics[width=0.21\textwidth]{./images/people_17-19th_century/Hermann_von_Helmholtz.jpg}}



\vspace{-.3cm}
\center{\scriptsize 
\href{https://en.wikipedia.org/wiki/William_Thomson,_1st_Baron_Kelvin}{Kelvin (1824-1907)}
\hspace{1cm} circulation \&  vorticity
\hspace{2cm}
\href{https://en.wikipedia.org/wiki/Hermann_von_Helmholtz}{Helmholtz (1821-1894)}
}

\end{center}

\begin{exampleblock}{}
\begin{itemize}

\item Circulation is related to vorticity via Stokes' Theorem 
\begin{equation}
  C = \ointctrclockwise_{\partial S} {\bm v} \cdot \mathrm{d}{\bm r}   
 \; \; \; \;  \myeqq \; \; \; \;
 \int_{S} (\nabla \wedge {\bm v})  \cdot \hat{\bm n} \, \mathrm{d}S 
= \int_{S} {\bm \omega} \cdot \hat{\bm n} \, \mathrm{d}S.
\end{equation}

\item Kelvin's Circulation Theorem for inviscid baroclinic flow: 
\begin{equation}
  \frac{\mathrm{D}C}{\mathrm{D}t} 
= \int_{S}  \underbrace{ \rho^{-2} \, ( \nabla \rho \wedge \nabla p ) }_{\mbox{\tiny baroclinicity}} \cdot \hat{\bm n} \, \mathrm{d}S.
\end{equation}
 

\end{itemize}
\end{exampleblock}{}

\end{frame}



\begin{frame}
  \frametitle{Animation: Kelvin-Helmholz instability leading to turbulent mixing}

\begin{center}
  \href{https://vimeo.com/187121815}{Computer simulation of
    Kelvin-Helmholz instability (shear instability)} using code from
  the Dedalus Project (an open source partial differential equation
  solver using spectral methods).
\end{center}

\end{frame}



\begin{frame}
  \frametitle{Maxwell and Gibbs: Thermodynamics}

\begin{center}
\vspace{-.25cm}
\fbox{\includegraphics[width=0.25\textwidth]{./images/people_17-19th_century/James_Clerk_Maxwell.png}}
\hspace{1cm}
\fbox{\includegraphics[width=0.30\textwidth]{./images/schematics/Maxwell's_letters_plate_IV.jpg}}
\hspace{1cm}
\fbox{\includegraphics[width=0.22\textwidth]{./images/people_17-19th_century/Josiah_Willard_Gibbs.jpg}}

\vspace{-.3cm}
\center{\scriptsize 
\href{https://en.wikipedia.org/wiki/James_Clerk_Maxwell}{Maxwell (1831-1879)}
\hspace{1cm}  Maxwell: $H_{2}O$ P-V-T phase diagram
\hspace{1cm}
\href{https://en.wikipedia.org/wiki/Josiah_Willard_Gibbs}{Gibbs (1839-1903)}
}

\end{center}

\begin{exampleblock}{}
\begin{itemize}
\item Thermodynamics provides relations between energy, entropy,
  enthalpy, heat, work, etc.

\item Developed a geometric view of thermodynamic phase spaces.  

\end{itemize}
\end{exampleblock}{}

\end{frame}



\begin{frame}
  \frametitle{McDougall: seawater thermodynamics}

\begin{center}
\vspace{-.25cm}
\fbox{\includegraphics[width=0.35\textwidth]{./images/people_20th_century_GFD/Trevor-McDougall.jpg}}
\hspace{2cm} 
\fbox{\includegraphics[width=0.32\textwidth]{./images/schematics/helicalA.pdf}}

\vspace{-.3cm}
\center{\scriptsize 
\href{https://en.wikipedia.org/wiki/Trevor_McDougall}{McDougall (UNSW professor)}
\hspace{3cm} Neutral helicity
}

\end{center}

\begin{exampleblock}{}
\begin{itemize}

\item Applications of thermodynamics to seawater were surprisingly
  simplistic until the seminal work of McDougall and colleagues during
  20th and 21st centuries.

\item Pioneered studies into the geometry and topology of seawater
  thermodynamics and ocean mixing.

\end{itemize}
\end{exampleblock}{}
\end{frame}



\begin{frame}
  \frametitle{Bjerknes, Rossby, Charney: \\ geophysical fluid mechanics}

\begin{center}
\vspace{-.25cm}
\fbox{\includegraphics[width=0.15\textwidth]{./images/people_20th_century_GFD/Vilhelm_Bjerknes_Bust_01.jpg}}
\hspace{1cm} 
\fbox{\includegraphics[width=0.15\textwidth]{./images/people_20th_century_GFD/Carl-Gustav-Rossby.jpg}}
\hspace{1cm} 
\fbox{\includegraphics[width=0.3\textwidth]{./images/people_20th_century_GFD/Charney2_0.jpg}}

\vspace{-.3cm}
\center{\scriptsize 
\hspace{-.4cm}
\href{https://en.wikipedia.org/wiki/Vilhelm_Bjerknes}{Bjerknes (1862-1951)}
\hspace{.3cm}
\href{https://en.wikipedia.org/wiki/Carl-Gustaf_Rossby}{Rossby (1898-1957)}
\hspace{1cm}
\href{https://en.wikipedia.org/wiki/Jule_Gregory_Charney}{Charney (1917-1981)}
}

\end{center}

\begin{exampleblock}{}
\begin{itemize}

\item Bjerknes: atmospheric motion can be described using laws of
  thermodynamics + fluid mechanics.

\item Rossby: wave patterns seen on weather maps arise from
  latitudinal dependence of the Coriolis force found on a spherical
  planet: {\it Rossby waves}.

\item Charney (also Eady): {\it baroclinic instability} = hydrodynamic
  instability feeding off available potential energy of a rotating
  stratified fluid: space-time scale for weather patterns; ocean
  eddies; macro-turbulence.  


\end{itemize}
\end{exampleblock}{}
\end{frame}




\begin{frame}
  \frametitle{Ocean mechanics = thermo-fluid mechanics on a rotating
    \& gravitating sphere}

\begin{center}
\vspace{-.25cm}
{\includegraphics[width=0.35\textwidth]{./images/schematics/rotating_sphere.pdf}}
\end{center}


\footnotesize 
\begin{align}
  \frac{\mathrm{D} {\bm v}  }{\mathrm{D}t} 
  + 2  \, {\bm \Omega} \wedge {\bm v} &= 
  - g \, \hat{\bm z} -\rho^{-1} \,  \nabla p + {\bm F}
 & \mbox{momentum (dynamics)}  
\label{eq:gfd-fluid-equation-of-motion-summary} 
  \\
  \frac{\mathrm{D}\rho}{\mathrm{D}t} &= -\rho \, \nabla \cdot {\bm v}
 &  \mbox{continuity (kinematics)}  
\label{eq:gfd-fluid-mass-equation-summary} 
  \\
 \rho \, \frac{\mathrm{D}\theta}{\mathrm{D}t}
 &= -\nabla \cdot {\bm J}(\theta)
 & \mbox{enthalpy (thermodynamics)}
  \label{eq:gfd-fluid-thermo-equation-summary} 
  \\
 \rho \, \frac{\mathrm{D}S}{\mathrm{D}t} 
 &= -\nabla \cdot {\bm J}(S)
 & \mbox{salinity (matter)}
\label{eq:gfd-fluid-matter-equation-summary} 
 \\
   \rho &= \rho(S,\theta,p). 
 & \mbox{equation of state (thermodynamics)}
\label{eq:eos-summary}   
\end{align}

\end{frame}


\begin{frame}
  \frametitle{Foundations for general circulation models}

\begin{center}
\vspace{-.25cm}
\fbox{\includegraphics[width=1.0\textwidth]{./images/people_20th_century_GFD/20th_century_icons.pdf}}
\end{center}

\begin{exampleblock}{}
\begin{itemize}

\item Developed numerical methods to turn the continuum partial
  differential equations into discrete algebraic equations suitable
  for computer simulations.

\item Developed {\it filtered} equations that target waves and eddying
  patterns relevant for weather and {\it general circulation}.

\item Exposed the chaotic nature of weather and macro-turbulence (chaos theory). 

\item Computer models provide an experimental tool to study the
  atmosphere and ocean.

\end{itemize}
\end{exampleblock}{}

\end{frame}



\section{Emergent patterns and fluctuations}


\begin{frame}
  \frametitle{Animation: fluid dynamics in the Southern Ocean}

\begin{center}
\vspace{-.25cm}
\fbox{\includegraphics[width=0.7\textwidth]{./images/pictures/NCRIS-Raijin2-3.jpg}}
\end{center}

\begin{center}
Computer model simulation of flow regimes in the Southern Ocean from the
\href{https://www.youtube.com/watch?v=8VMSF28J9H4}{Australian ACCESS-OM2 0.1 degree ocean/sea-ice model} 
run on the NCI Raijin supercomputer in Canberra, AUS.
\end{center}

\end{frame}


\begin{frame}
  \frametitle{There's a zoo of physical ocean processes}

\begin{center}
\vspace{-.4cm}
{\includegraphics[width=0.45\textwidth]{./images/schematics/Figure_Malou_June27_exportJPEG.jpg}}
\end{center}
\vspace{-.5cm} 
\center{\scriptsize \href{https://www.annualreviews.org/doi/abs/10.1146/annurev-marine-010318-095421}{Groeskamp et al (2019)}}

\vspace{0cm}
\begin{exampleblock}{}
\begin{itemize}

\item The ocean is a forced-dissipative system with cascades of energy
  across space-time scales.

\item Enhanced resolution of observations (e.g., satellite) and
  simulations generally reveals refined layers of phenomenology.
   \begin{itemize}
      \item[$\star$] The more we look, the more we see! 
   \end{itemize}

\item Direct implications for representation and parameterization of
  the ocean's role in climate and ecosystems. 
 
\end{itemize}
\end{exampleblock}{}


\end{frame}




\begin{frame}
  \frametitle{Space-time diagram of ocean dynamical processes}

\begin{columns}[c] 
\column{.5\textwidth} 
\begin{center}
\vspace{-.25cm}
\fbox{\includegraphics[width=0.9\textwidth]{../../../../../Documents/OTHERS/USA/OREGON/chelton/research_hotswg_HOTSWG_overview_space_time_schematic.pdf}}
\center{\scriptsize \href{http://www-po.coas.oregonstate.edu/research/po/research/hotswg/HOTSWG_report.pdf}{Chelton (2001)}}
\end{center}

\column{.50\textwidth}
\begin{exampleblock}{}
\begin{itemize}

\item Broad range of space-time scales.

\item Absence of a clear spectral gap a result of nonlinear cascade of
  turbulent flows across scales. 
   \begin{itemize}
      \item[$\star$] There is no desert.  
   \end{itemize}

\item The absence of a spectral gap introduces some difficult
   problems to the turbulence closure problem in geophysical flows.

\end{itemize}
\end{exampleblock}{}

\end{columns}
\end{frame}


\begin{frame}
  \frametitle{Macro-scale turbulence: mesoscale + submesoscale}

\begin{center}
\vspace{-.45cm}
{\includegraphics[width=0.90\textwidth]{./images/eddies/MODIS_and_ROMS_zeta_over_f.pdf}}
\end{center}
\vspace{-.6cm} 
\center{\scriptsize \href{https://oceancolor.gsfc.nasa.gov/gallery/609/}{Phytoplankton bloom from NASA GSFC}
      \hspace{1cm}  \href{https://journals.ametsoc.org/doi/10.1175/JPO-D-14-0154.1}{Simulation (Gula et al 2015) ($\Delta = 500$m)}}


\begin{exampleblock}{}
\begin{itemize}

\item Satellite measurements reveal the multi-scale structure of ocean
  macro-turbulence.

\item Simulations allow us to study the mechanisms.
   \begin{itemize}
   \item[$\star$] Primary eddy features (``mesoscale eddies'')
     initiate secondary ``submesocale instabilities'' that lead to
     filaments along the edges of the eddies.
   \item[$\star$] Submesoscale eddies and fronts have strong vertical
     motions that bring deep nutrients into the upper photic (light)
     zone.
   \end{itemize}

\end{itemize}
\end{exampleblock}{}

                           

\end{frame}



\begin{frame}
  \frametitle{Coherent structures + turbulent soup = order in chaos}

\begin{center}
\vspace{-.3cm}
{\includegraphics[width=0.4\textwidth]{./images/eddies/Kuroshio_LAVD.pdf}}
\hspace{1cm}
{\includegraphics[width=0.4\textwidth]{./images/eddies/SouthernOceanSpeed.pdf}}
\vspace{-.3cm}
\center{\scriptsize 
       \href{https://www.sciencedirect.com/science/article/pii/S1463500318302348}{LAVD field from Tarshish et al (2018)}
      \hspace{1cm} Surface current speed (courtesy A. Morrison, ANU) }

\end{center}

\begin{exampleblock}{}
\begin{itemize}
\item We see coherent structures within a background turbulent soup. 
\begin{itemize}
\item[$\star$] Quasi-geostrophic turbulence cascades energy to large
  scale.
\item[$\star$] In some regions the cascade focuses on coherent
  structures: \\  \href{https://vimeo.com/213705279}{Cascade animation from R. Abernathey, Columbia Univ.}
\item[$\star$] Maximize information entropy while conserving energy.

\end{itemize} 

 \item Non-autonomous dynamical systems methods to objectively
   identify coherent structures and measure impacts on transport.
\end{itemize}
\end{exampleblock}{}

\end{frame}



\section{Climate change and sea level}


\begin{frame}
  \frametitle{Earth is warming due to fossil fuel burning (IPCC)}

\begin{center}
\vspace{-.25cm}
{\includegraphics[width=1\textwidth]{./images/climate/global_warming.pdf}}
\vspace{-.75cm}
\center{\scriptsize \href{https://climate.nasa.gov/vital-signs/global-temperature/}{NASA analysis}}
\end{center}

\begin{exampleblock}{}
\begin{itemize}
\item Consistent pattern of global and regional warming of surface temperatures. 
\item Each season is warming.
\item Northern high latitudes are warming the most, reflecting
  ice-albedo feedback as a result of melting Arctic sea-ice.
\item North Atlantic cooling might reflect slow-down of Atlantic
  overturning circulation.
\end{itemize}
\end{exampleblock}{}

\end{frame}


\begin{frame}
  \frametitle{Global warming = ocean warming $\Rightarrow$ sea level rises}

\begin{center}
\vspace{-.25cm}
{\includegraphics[width=.70\textwidth]{./images/climate/sealevel_and_ocean_warming.pdf}}
\vspace{-0.5cm}
\center{\scriptsize \href{https://ar5-syr.ipcc.ch/topic_observedchanges.php}{IPCC AR5}
\hspace{2cm} 
 \href{https://climate.nasa.gov/vital-signs/global-temperature/}{NASA analysis}
}
\end{center}

\vspace{-.3cm} 
\begin{exampleblock}{}
\begin{itemize}
\item Ocean warming raises sea level (thermosteric + land ice melt);
  $\approx 30$ cm since 1880 and 3.2 mm/yr since 1993 (accelerating).
\item Regional patterns associated with ocean dynamical processes:
  winds, regional heating, changes to circulation.
\item Since 1970, ocean is heating at a rate equivalent to the power
  released by roughly 3 to 6 Hiroshima bombs per second.
\end{itemize}
\end{exampleblock}{}

\end{frame}


\begin{frame}
  \frametitle{Winds, waves, and warming Antarctic ice shelves}


\begin{center}
\vspace{-.35cm}
{\includegraphics[width=.7\textwidth]{./images/climate/Antarctic_sealevel_processes.pdf}}
\vspace{-0.25cm}
\center{\scriptsize \href{http://science.sciencemag.org/content/348/6232/327}{Paolo et al (2015)}
\hspace{2cm} 
 \href{https://www.nature.com/articles/nclimate3335}{Spence et al (2017)}}
\end{center}

\vspace{-0.25cm}
\begin{exampleblock}{}
\begin{itemize}
\item Complex array of physical processes at the base of ice shelves.  
\item Broad trends in ice shelf melting over recent 20 years. 
\item Models and theory suggest that winds and waves can act to alter
  interior pressure gradients that in certain regions lead to movement
  of warm off-shore water towards ice shelves.

\end{itemize}
\end{exampleblock}{}

\end{frame}


\begin{frame}
  \frametitle{Further big questions of ocean climate science}

\begin{center}
\vspace{-.3cm}
{\includegraphics[width=0.65\textwidth]{./images/climate/BigClimateQuestions.pdf}}
\vspace{-0.25cm}
\center{\scriptsize 
\href{https://journals.ametsoc.org/doi/10.1175/BAMS-D-11-00151.1}{Srokosz (2012)}
\hspace{0.5cm} 
\href{https://www.nature.com/articles/s41586-018-0006-5?WT.feed_name=subjects_climate-and-earth-system-modelling}{Caesar et al (2018)}
\hspace{0.5cm} 
\href{https://physicstoday.scitation.org/doi/10.1063/PT.3.2654}{Morrison et al (2015)}
\hspace{0.5cm} 
\href{https://www.britannica.com/science/ocean-acidification}{Encyclopaedia Britannica}
}
\end{center}

\vspace{-0.25cm}
\begin{exampleblock}{}
\begin{itemize}

\item Slowdown of Atlantic overturning circulation, with impacts on
  poleward/vertical heat and carbon transport, European climate, and
  North American east coast sea level;

\item Southern Ocean ventilation of heat and carbon with possible
  changes due to changing winds, precipitation, and heating;

\item Ocean heat waves, acidification, and impacts on marine life.

\end{itemize}
\end{exampleblock}{}

\end{frame}





\section{Closing comments}

\begin{frame}
  \frametitle{Summary}

\begin{exampleblock}{}

\begin{itemize}

\item Ocean circulation emerges from the underlying thermo/mechanical
  equations for a fluid on a gravitating and rotating sphere.

\item Ocean circulation studies involve a synergy between observations
  (remote and {\it in situ}), theory (maths, physics), and
  simulations (numerics, computer science).

\item Simulations provide an experimental tool to reveal a mechanistic
  understanding of the complex multi-scale fluid flow.

\item Civilization is presenting the earth with a (relatively new)
  geophysical force whose impacts are felt by the ocean and
  atmosphere: 
   \begin{itemize}
   \item[$\star$]  The ``great carbon burning experiment'' 
   \item[$\star$] Humanity does not control climate but we do affect
     it.
   \item[$\star$] Unraveling the mechanisms for both natural and
     anthropogenic fluctuations in the ocean/atmosphere climate system
     is fundamental to 21st century earth science.
  \end{itemize}
 
\end{itemize}


\end{exampleblock}{}

\end{frame}



\begin{frame}
  \frametitle{A suite of further cutting-edge animations}

\begin{itemize} 

\item[$\star$] 
\href{https://vimeo.com/dedalus}{Process studies using Dedalus}

\item[$\star$] 
\href{http://stockage.univ-brest.fr/~gula/movies.html}{Multi-scale flows from J. Gula using ROMS}

\item[$\star$] 
\href{https://vimeo.com/oceannext}{Atlantic simulations from NEMO at $<2$km grid spacing}

\item[$\star$] 
\href{https://www.youtube.com/watch?v=CCmTY0PKGDs}{NASA global surface circulation using MITgcm}


\end{itemize}

\end{frame}



\begin{frame}
  \frametitle{Some words for the aspiring mathematical scientist}


\begin{center}
\vspace{-.4cm}
{\includegraphics[width=0.55\textwidth]{./images/pictures/quotes_pics.pdf}}
\end{center}

\vspace{-.4cm}

\begin{exampleblock}{}

\small 
\begin{itemize} 

\item[$\star$] {\sc Albert Einstein:} The important thing is not to
  stop questioning. Curiosity has its own reason for existence.

\item[$\star$] {\sc R. Buckminster Fuller:} You never change things by
  fighting the existing reality. To change something, build a new
  model that makes the existing model obsolete.

\item[$\star$] {\sc Gautama Buddha:} Do not believe in anything simply
  because you have heard it.  Do not believe in anything merely on the
  authority of your teachers and elders. But after observation and
  analysis, when you find that anything agrees with reason and is
  conducive to the good and benefit of one and all, then accept it and
  live up to it.


\item[$\star$] {\sc R.B. Ginsburg:} Fight for the things that you care
  about, but do it in a way that will lead others to join you.


\end{itemize}

\end{exampleblock}{}

\end{frame}


\begin{frame}
\frametitle{Many thanks for your time and attention}

\begin{center}
 It has been an honour and pleasure to present this lecture. 
\vspace{-.25cm}
\fbox{\includegraphics[width=1.0\textwidth]{./images/pictures/Southern_Ocean_photos.pdf}}
\end{center}
\center{\scriptsize From the Weddell Sea and Scotia Sea, autumn 2017 on the RRS James Clark Ross}


\end{frame}  


\end{document}

