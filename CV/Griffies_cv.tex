\documentclass{article}
\usepackage{geometry}\geometry{top=1.5cm,left=1.cm,right=1.0cm,bottom=1.75cm,centering,dvips}
\usepackage{color}
\usepackage[usenames,dvipsnames]{xcolor}
\usepackage{shadow}
\usepackage{kpfonts}
\usepackage{enumitem}
\usepackage{lastpage}
\usepackage{bm}
\usepackage{verbatim}
\usepackage{graphicx}
\usepackage{hyperref}
\hypersetup{colorlinks=true,linkcolor=OliveGreen,citecolor=blue,filecolor=blue,urlcolor=blue} 

\usepackage[british]{babel}
\usepackage[yyyymmdd]{datetime}

\usepackage{etaremune}

% Header box
\def\name#1{\def\@name{#1}}
\def\info#1{\def\@info{#1}}
\newcommand{\shadebox}[3][.9]{\fcolorbox[gray]{0}{#1}{\parbox{#2}{#3}}}
\def\maketitle{
\thispagestyle{plain}
\vspace*{-1.5cm}
\shadebox[0.9]{17.3cm}{\sf\color[rgb]{.6,0,0}
}
\vspace*{0.2cm}
}

\usepackage[Lenny]{fncychap}
\usepackage{fancyhdr}
\pagestyle{fancy}
\fancyhead{}
\fancyfoot{}

\chead{\sc \color{Maroon} \protect\iconB{waves.pdf} Stephen M. Griffies \protect\iconB{waves.pdf} }

\cfoot{\sc  \color{Maroon}  Curriculum Vitae}
\lfoot{  \color{Maroon} \today}
\rfoot{\color{Maroon} \sc page  \thepage\ of \pageref{LastPage}}
\renewcommand{\footrulewidth}{1pt}

% icons for images 

\newcommand{\icon}[1]
{\includegraphics[height=.015\textheight]{#1}}

\newcommand{\iconB}[1]
{\includegraphics[height=.012\textheight,width=0.075\textwidth]{#1}}


\newcommand{\largeicon}[1]
{\includegraphics[height=.015\textheight,width=0.5\textwidth]{#1}}

\fancyhead[LE,LO]{\textsc{\textbf{\nouppercase{\leftmark}}}}

\begin{document}
\thispagestyle{empty}

\shadebox[0.9]{18cm}{\sf\color[rgb]{.6,0,0}
\begin{center}

\sc 

%\begin{center}
%\resizebox{3cm}{!}{\includegraphics{Griffies_Coronation_Island_2017_crop.jpg}}

\resizebox{3.0cm}{!}{\includegraphics{Griffies_photo_2021.jpeg}}

%\end{center}

\textsc{\large \color{Maroon} \protect\icon{spiral.pdf} Stephen Matthew Grif\/f\/ies (he/him/his)}
\protect\icon{ripple_spiral.pdf} 
\\[1ex]

%NOAA Geophysical Fluid Dynamics Laboratory (GFDL)
%$\bullet$ Princeton USA
%\\
Princeton University Atmospheric and Oceanic Sciences Program 
\\
{\tt \small  SMG at Princeton.edu} 
\hspace{.1cm}
$\bullet$ 
\hspace{.1cm} 
{\tt \small Stephen.M.Griffies at gmail.com} 
%$\bullet$ {\tt  \small Stephen.M.Griffies at gmail.com} 
\\  
\vspace{0.2cm}

\href{https://StephenGriffies.github.io/}{research and teaching \hspace{0.5cm}}
\href{https://orcid.org/0000-0002-3711-236X}{orcid \hspace{0.5cm}}
\href{https://scholar.google.com/citations?user=4LPPPBAAAAAJ&hl=en}{google scholar \hspace{0.5cm}}
\href{https://www.webofscience.com/wos/author/record/55160}{web of science}


\end{center}
}

\section*{\sc \color{Maroon}  research statement}
\vspace{-.3cm}
I pursue research questions related to geophysical fluid mechanics and the role of the physical ocean in the earth climate system. In pursuit of this research, I make use of theoretical concepts, idealized process physics models, realistic numerical circulation models, and field measurements. Particular research topics include studies of Atlantic and Southern Ocean circulation dynamics; global and regional sea level; transport of matter and energy by mesoscale and submesoscale eddies; subgrid scale parameterizations of turbulent ocean stirring and mixing; analysis methods aimed at conceptually understanding ocean circulation and transport; physical and mathematical foundations of ocean circulation models; fundamental concepts and methods of geophysical fluid mechanics. 

\vspace{-.5cm}
\section*{\sc \color{Maroon} education statement}
\vspace{-.3cm}
As a teacher, mentor, author, and journal editor, I aim to foster an understanding of physical concepts and their creative and rigorous use in describing observed and simulated ocean and climate phenomena. I am particularly interested in revealing how concepts and methods from mathematical physics can be leveraged to deepen understanding of earth climate mechanics, and for fostering an appreciation of geophysical fluid mechanics within the broader context of theoretical mechanics, thermodynamics, and mathematical physics. Teaching and mentoring are central to all of my work, and I strive to generate curiosity and passion in colleagues while fostering an honest and non-judgmental scientific pursuit of how nature works. 

\vspace{-.5cm}
\section*{\sc  \color{Maroon}  collaboration statement}
\vspace{-.3cm}
I nurture collaborations with scientists at all career stages who are passionate about furthering a rigorous understanding of ocean and climate physics. 

\vspace{-.5cm}
\section*{\sc  \color{Maroon}  broader interests and activities}
\vspace{-.3cm}
meditation, yoga, walking, writing, sustainability, cultures, surfing, skiing

\noindent\rule{\textwidth}{1pt}


\vspace{-.2cm}
\section*{\sc \color{Maroon} education}
\vspace{-.25cm}
\begin{tabular}{lll}
1995-1996  &  Postdoctoral fellow in physical oceanography (mentor: Kirk Bryan) & Princeton University 
\\
1993-1995  &  NOAA Climate and Global Change Fellow (mentor: Kirk Bryan) & Princeton University 
\\
1988-1993  &  PhD in theoretical physics 
(advisor: Mirjam Cveti\v{c}) 
& University of Pennsylvania 
\\
1987-1988  &  pre-PhD studies in physics & University of Washington
\\
1986-1987  &  Masters in engineering sciences \& applied mathematics   & Northwestern University\\
1981-1986  &  Bachelor of science in chemical engineering  & Louisiana State University \\
1978-1981  & High school (grades 10-12) & Biloxi High School, Mississsippi
 \end{tabular}


\vspace{-.2cm}
\section*{\sc \color{Maroon} special topic schools}
\vspace{-.25cm}
\begin{tabular}{lll}
Jan 1998  &  NATO Advanced Study Institute: {\sc Ocean Modeling and Parameterization} & Les Houches, France 
\\
Feb 1995 & NATO Advanced Study Institute: {\sc Climate Variability and Predictability} & Les Houches, France
\\
Jul 1994 & Meeting of UCAR Global and Climate Change Fellows & Steamboat Springs, Colorado, USA
\\
Jul 1992 & Theoretical Advanced Study Institute: {\sc From String Theory to Black Holes} & Boulder, Colorado, USA
\\
Jul 1991 & High Energy Physics and Cosmology School, Center for Theoretical Physics & Trieste, Italy
\\
Jun 1991 & Theoretical Physics Summer School: {\sc Particle Physics in the 1990's} & Les Houches, France
 \end{tabular}


\vspace{-.5cm}
\section*{\sc \color{Maroon} employment and appointments} 
\vspace{-.25cm}
\begin{tabular}{ll}
  2024--2025 & Chair of the Visiting Scientist  Committee for the Cooperative Institute for Modeling the Earth System 
  \\
  
   2024--2025 & Director of Graduate Studies for Princeton University's Atmospheric and Oceanic Sciences Program
  \\
  
  2022--2024 & Graduate Work Committee for Princeton University's Atmospheric and Oceanic Sciences Program
  \\
  
  2020--2024 & Team lead for the GFDL Ocean/Cryosphere Division's high resolution climate model project CM4X 
  \\

  2015--present & Lecturer in Princeton University's Atmospheric and Oceanic Sciences Program
  \\
 
  2013--2017  & Member, GFDL Model Development Team Steering Committee  \\
  
  Jun-Aug 2012  & Visiting Scientist, National Center for Atmospheric
                  Research, Boulder, USA \\

  Jan-Jun 2011   & Distinguished Visiting Scientist Fellow, CSIRO Marine and Atmospheric Research, Hobart, Australia \\
 
  Mar 2009         & Visiting Professor, Universite catholique de Louvain, Belgium \\
 
  Jan-Nov 2005   & Visiting Scientist, CSIRO Marine and Atmospheric  Research, Hobart, Australia \\
 
  2001--2005     & Leader of the GFDL Oceans and Climate Group \\
 
  2000--2011     & Co-lead of the GFDL Ocean Model and Climate Model Development Teams \\
  
  1996-2025   &  Staff Physical Scientist, NOAA/GFDL (senior scientist 2011-2025, retired 1 May 2025)  \\  
  
  1995--1996     &  Visiting Scientist, GFDL and Princeton University  \\ 
  
  1993--1995     & NOAA Climate \& Global Change Postdoc Fellow at Princeton University \\
  
  1988--1993     &  Physics Graduate Research Fellow, University of Pennsylvania Physics Department  \\                     
  
  1986--1987     &  Engineering Sciences and Applied Mathematics Fellow, Northwestern University \\
  
  1984--1986     &  Chemical Engineering Research Laboratory Technician, Louisiana State University

\end{tabular}


\noindent\rule{\textwidth}{1pt}
\vspace{-.5cm}
\section*{\sc  \color{Maroon}   awards and honors}
\vspace{-.25cm}

\begin{tabular}{ll}
  2022 & NOAA Administrator's Award (with 26 others) "For advancing the understanding of the Earth System \\ & by developing and applying NOAA's state-of-the-art Coupled Carbon-Chemistry-Climate model"
\\
  2021 & \href{https://app.dimensions.ai/details/entities/publication/author/ur.010052126545.52}{Reuters Hot List of Climate Scientists (\#585)} 
  \\
  2019 & Department of Commerce Silver Medal Award (with Robert Hallberg and Matthew Harrison): "For \\ &  developing the state-of-the-art Modular Ocean Model version 6 (MOM6) to strengthen the Nation's \\& longer-range environmental prediction capabilities."
  \\
  2019 & Sigma Xi scientific honor society 
  \\
  various  & Web of Sciences (Clarivate) Highly Cited Researcher (2018, 2020, 2021, 2022, 2023, 2024)
  \\
  2017 & \href{https://eos.org/agu-news/celebrating-the-2017-class-of-fellows}{Elected Fellow of the American Geophysical Union} "For exceptional and sustained contributions to the \\ &  understanding of large-scale ocean circulation and physics and seminal advances in ocean modeling"
\\
  2017 & NOAA Administrator's Award (with Robert Hallberg) "For scientific leadership for the innovation \\ & of the versatile  community-based Modular Ocean Model MOM6" 
  \\
  2014 & \href{http://www.egu.eu/awards-medals/fridtjof-nansen/2014/stephen-m-griffies/}{European Geosciences Union Fridtjof Nansen Medal for
         Oceanographic Research}  "For 
outstanding \\ & contribution and leadership in 
ocean general circulation model development 
and critical insights in the \\ & physical 
nature and parameterization of ocean processes"
\\
  2013 & Department of Commerce Silver Medal Award (with nine other
  GFDL staff scientists): 
  "For development \\ & and application of NOAA's first comprehensive  
  Earth System Model  
  that couples the carbon cycle and \\ & climate for projection of changes" \\
  2012 & NOAA Administrator's Award "For scientific vision, leadership
  and development of 
  the Modular Ocean \\ & Model (MOM4) for climate modeling, research and
  predictions" \\
  2011 & CSIRO Distinguished Visiting Scientist Fellow, Australia \\
  2009 & Visiting Professor, Universite catholique de Louvain, Belgium\\
  2001 & NOAA/Oceanic and Atmospheric Research Outstanding Scientific
  Review Paper\\
  1999 & NOAA/Oceanic and Atmospheric Research Outstanding Scientific Paper\\
  1998 & NOAA/Oceanic and Atmospheric Research Employee of the Year\\
  1997 & NOAA/Environmental Research Laboratories Outstanding Scientific Paper\\
  1993-1995 & NOAA Climate and Global Change Research Fellow
\end{tabular}


\noindent\rule{\textwidth}{1pt}
\vspace{-1cm}
\section*{\sc  \color{Maroon}  professional services and memberships}
\vspace{-.25cm}

\begin{tabular}{ll}
    2024--present & Scientific advisory board for the C-Star program at [C]Worthy
  \\
2021-present & NEMO Scientific Advisory Committee
\\

2021-present & Editor-in-Chief for  AGU's  Journal of Advances in Modeling Earth Systems (JAMES) 
 \\
% 2020-2022 & Princeton University AOS Diversity, Equity, Inclusion, and Accessibility committee
% \\
2018-2020 & Ocean/Cryosphere Editor for AGU's  
Journal of Advances in Modeling the Earth System (JAMES) 
\\
2019-2023 & Chair of the awards committee for the EGU Fridtjof Nansen Medal for Oceanographic Excellence
\\
2017-2021 & Advisory board for the TICTOC Project, United Kingdom
\\
  2016-2019 & Awards committee for the EGU Fridtjof Nansen Medal for Oceanographic Excellence 
  \\
2014-2018 &  WCRP/CLIVAR Scientific Steering Group 
\\
2014-2016     & Co-lead of the NCEP Climate Model Development Task Force
\\
2012-2014     & CLIVAR/CliC/SCAR Southern Ocean Region Implementation Panel 
\\
2012-present &  WCRP/CLIVAR Ocean Model Development Panel (ex officio)
\\
2010-present & European Geosciences Union (EGU)
\\
2009-2015  &  Scientific Advisory Board for the Catalan  Climate Institute {\it IC3}, Barcelona, Spain 
\\
2007-2018 & Editor of the journal {\bf Ocean Modelling} 
\\
2006-2009     &  WCRP/CLIVAR Scientific Steering Group (ex officio) 
\\
2004-2009     &  WCRP/CLIVAR Working Group on Coupled Modelling (ex officio) \\

2004-2007     & Editorial Board of the journal {\bf Ocean Science} \\

1999-2012     & WCRP/CLIVAR Working Group on Ocean Model Development  (co-chair 2004-2009) \\

1993-present  & American Geophysical Union and American Meteorological Society \\

\end{tabular}

\noindent\rule{\textwidth}{1pt}
\vspace{-1cm}
\section*{\sc  \color{Maroon}  university teaching}
\vspace{-.3cm}

\begin{itemize}[leftmargin=*]

\item Autumn semester 2025 (planned): Princeton University Atmospheric and Oceanic Sciences 580: Special topics: {\bf Methods in the Analysis of Ocean Scalar Properties}. 24 lectures of 90 minutes duration covering the full course, presenting a suite of theoretical tools to support physical ocean analysis. Topics include material tracers, ocean thermodynamics, ocean energetics, tracer advection and diffusion, turbulent tracer transport, subgrid scale parameterizations, Green's function methods, watermass transformation analysis. Lecture material base on \href{https://stephengriffies.github.io/assets/pdfs/GFM_lectures.pdf}{Griffies (2025): Geophysical Fluid Mechanics.}

\item Spring semester 2023, 2024, 2025: Princeton University Atmospheric and Oceanic Sciences 572: {\bf Theory of Geophysical Fluid Waves and Instabilities}. 24 lectures of 80-90 minutes duration covering the full course, presenting theoretical foundations for ocean and atmosphere linear wave mechanics and instability theory. Topics include wave kinematics, acoustics, capillary waves, surface gravity waves, inertial waves, Rossby waves, shallow water waves, internal inertia-gravity waves, shear instability, inertial instability, symmetric instability, baroclinic instability. Lecture material base on \href{https://stephengriffies.github.io/assets/pdfs/GFM_lectures.pdf}{Griffies (2025): Geophysical Fluid Mechanics.}

\item Autumn semester 2014 (0.5), 2015 (0.5), 2016 (0.5), 2017, 2018, 2019, 2020, 2021, 2022, 2023, 2024 (0.5): Princeton University Atmospheric and Oceanic Sciences 571: {\bf  Foundations of Geophysical Fluid Mechanics}. 24 (12) lectures of 80-90 minutes duration covering the full (half) of course, focusing on the theoretical foundations for geophysical fluid mechanics. Topics include mechanics of motion on a rotating planet, Eulerian and Lagrangian fluid kinematics, frictional stresses, pressure and form stress, buoyancy, Ekman mechanics, shallow water mechanics, vorticity and potential vorticity, quasi-geostrophy. Lecture material base on \href{https://stephengriffies.github.io/assets/pdfs/GFM_lectures.pdf}{Griffies (2025): Geophysical Fluid Mechanics.}

\item Spring semester 2020: Princeton University Atmospheric and Oceanic Sciences 521: {\bf Southern Ocean Seminar}. I provided five 90-minute lectures covering Southern Ocean dynamics, while other lecturers presented allied material to cover Southern Ocean science.  Lecture material base on \href{https://stephengriffies.github.io/assets/pdfs/GFM_lectures.pdf}{Griffies (2025): Geophysical Fluid Mechanics.} 

\item Spring semester 2017, 2018, 2019, 2024: Princeton University Atmospheric and Oceanic Sciences 580: {\bf Special Topics on Great Papers in Atmospheric and Oceanic Sciences}. I led on 90-minute discussion session on a chosen classic paper in ocean fluid mechanics. 

\item Spring semester 2016, 2019: Princeton University Geosciences 503: {\bf Responsible Conduct of Research in Geosciences}.  I co-taught one three-hour discussion session on ethical behavior in research. 

\item Autumn semester 1993: Princeton University Atmospheric and Oceanic Sciences Special Topics 580: {\bf Data Assimilation in Atmospheric and Oceanic Models}. I was the co-lecturer and coordinator of visiting lectures throughout the semester. 

\item 1990--1993:  Instructor, Undergraduate Physics Laboratory, University of Pennsylvania 

\item 1990--1993:  Teaching Assistant,  General Relativity and Quantum Field Theory, University of Pennsylvania 

\end{itemize}



\noindent\rule{\textwidth}{1pt}
\vspace{-.5cm}
\section*{\sc  \color{Maroon} mentoring and sabbatical hosting}
\vspace{-.25cm}

\begin{tabular}{lll}

2024 & Claire Yung & visiting graduate student (from ANU, Canberra, AUS) \\

2024-present & Stefan Kildal-Brandt& Princeton University graduate student (geophysics) \\

2023-present & Kiera Lowman& Princeton University graduate student (AOS)  \\

2023-present & Maxime Keutgen De Greef& Princeton University graduate student (AOS) \\

2023-present & Kentaro Hanson& Princeton University graduate student (applied mathematics) \\

2023 & Jan Zika & Princeton University visiting scholar (from UNSW, Sydney, AUS) 
\\

2022-present & Matthew Lobo& Princeton University graduate student (AOS, primary mentor) \\

2022-present & Winnie Chu& Princeton University graduate student (AOS) \\

2022-present & Wenda Zhang& Princeton University postdoc fellow \\

2021-2022 & Rachel Pang& Princeton University undergraduate student (junior paper mentor) \\ 

%2021-2022 & Marta Faulkner& Princeton University graduate student \\ 

2021 & Abigail Bodner& Brown University graduate student (PhD thesis reader) \\ 

%2020-present & Toyo Sadare& Princeton University graduate student (PhD committee) \\ 

2020-present & Jan-Erik Tesdal& Princeton University postdoc fellow \\ 

2020 & Ruth Moorman & Princeton University predoc intern \\ 

2019-2020 & Benjamin Taylor & Princeton University predoc intern \\ 

2019-2021 & Hemant Khatri & Princeton University postdoc fellow  \\ 

2019-2020 & Elizabeth Yankovsky & Princeton University graduate student (AOS) \\ 

2019     & Hussein Aluie & Princeton University visiting scholar (from University of Rochester)  \\ 

2018-2022 & Graeme MacGilchrist & Princeton University postdoc fellow \\ 

2017-2022 & Houssam Yassin & Princeton University graduate student (AOS, primary mentor) \\ 

2017-2018 & Laure Zanna  & Princeton University visiting scholar (from Oxford University)  \\

2017 & Jianjun Yin       & Princeton University visiting scholar (from University of Arizona)  \\

2016-2019 & Brandon Reichl       & Princeton University postdoc fellow  \\

2016-2018 & Nathaniel Tarshish & Princeton University predoc intern \\

2015-2017 & Amanda O'Rourke  & University of Michigan postdoc fellow (with Brian Arbic) \\

2015-2016    & Henri Drake             & Princeton University predoc intern (with Jorge Sarmiento) \\

2014-2017 & Anna FitzMaurice   & Princeton University PhD student (AOS) \\ 

2014-2015     & Ivy Frenger            & Princeton University postdoc fellow (with Jorge Sarmiento) \\

2014 & Magnus Hieronymus& Stockholm University graduate student (PhD opponent) \\ 

2013-2017 & Robert Nazarian    & Princeton University PhD student (AOS) \\ 

2013-2016     & Adele Morrison     & Princeton University postdoc fellow (with Jorge Sarmiento) \\

2013               & Terrence O'Kane   & GFDL visiting scholar from CSIRO Marine Research, Hobart, Australia \\

2012-2017     & Carolina Dufour   & Princeton University postdoc fellow (with Jorge Sarmiento)  \\

2012-2013     & Yalin Fan              & Princeton University postdoc fellow  \\

2011-2014     & Michael Bueti       & University of Rhode Island  PhD student (PhD committee) \\

2008-2011     & Michael Bates       & University of New South Wales PhD student (PhD committee) \\

2005-2009     & Andreas Klocker   & University of Tasmania  PhD student (PhD committee) \\

2003-2004     & {R\"{u}diger} Gerdes  & GFDL visiting scholar (from AWI, Bremerhaven, Germany) \\

2001-2002     & Harper Simmons   & GFDL postdoc fellow \\

1999-2002     & Shafer Smith         & Princeton University postdoc fellow    
\end{tabular}


\noindent\rule{\textwidth}{1pt}
\vspace{-1cm}
\section*{\sc \color{Maroon}  oceanographic field work}
\vspace{-.25cm}

%\begin{tabular}{l}
\begin{itemize}[leftmargin=*]
 \item 
 Mar-May 2017: Eight week cruise on the {\it RRS JC Ross}  to the Orkney Passage and Scotia Sea,
  as part of the
  Dynamics of the Orkney Passage Outflow (DynOPO) project. Principal Scientific Officer: Alberto Naveira Garabato. 
 \item 
  Jul 1993: Two week cruise on the {\it CCGS Hudson} to the Labrador Sea in support of  the WOCE Line AR7W Atlantic Circulation Experiment. Chief Scientist: John Lazier.
\end{itemize}
%%\end{tabular}

%\newpage 



\noindent\rule{\textwidth}{1pt}
\vspace{-1cm}
\section*{\sc \color{Maroon}  participant/collaborator on research grants and projects}
\vspace{-.3cm}

\begin{itemize}[leftmargin=*]

\item Partner Investigator for the Australian Research Council Centre of Excellence in Our Future Oceans (under review)

\item Partner Investigator for the Australian Research Council (2021-2025) Centre of Excellence in Antarctic Science (ACEAS), AU\$25M.

\item PI for an NSF subcontract with New York  University for NOAA's Climate Variability and Predictability Program Climate Process Team: {\it Ocean Transport and Eddy Energy} (2024-2025), \$150K.

{\it Diagnostics and Performance Metrics for Evaluating Ventilation Pathways and Interior Water Mass Properties in Ocean Models} (2020-2022), \$180K. 
 
\item Co-PI for NOAA's Climate Variability and Predictability Program Climate Process Team: {\it Ocean Transport and Eddy Energy} (2020-2024), \$770K.

\item Co-PI for NOAA's Climate Variability and Predictability Program (CVP) {\it Decadal Climate Variability and Predictability} proposal {\it Drivers of coastal sea level change along the eastern US} (2020-2023), \$200K.  

\item PI for DOE subcontract with Princeton University for the {\it Diagnostics and Performance Metrics for Evaluating Ventilation Pathways and Interior Water Mass Properties in Ocean Models} (2020-2022), \$180K. 

\item Co-PI for Australian Research Council Discovery Project (2019-2022): {\it Risks of rapid ocean warming at the Antarctic continental margin}. AU\$582K.

\item Co-PI for NOAA Modeling, Analysis, Predictions, and Projections Program (01Aug2018--31Jul2020): {\it Addressing Key Issues in CMIP6-era Earth System Models}. \$434K.
    
\item Program advisory board for the UK NERC funded project: {\it Transient tracer-based Investigation of Circulation and Thermal Ocean Change (TICTOC)} (2017-2021).

\item Partner Investigator for the Australian Research Council (2017-2023) Centre of Excellence for Climate Extremes, AU\$30M.
  
\item Co-PI for the Ocean Model Intercomparison Project (OMIP), which is part of the Coupled Model Intercomparison Project (CMIP6) (2016-2022).    

\item Co-PI for the Flux Anomaly Forcing Model Intercomparison Project (FAFMIP), which is part of the Coupled Model Intercomparison Project (CMIP6) (2016-2022).    

\item Co-PI for NOAA Modeling, Analysis, Predictions, and Projections Program (01Jul2016--30Jun2018): {\it Development toward NCEP's fully-coupled global forecast and data assimilation system: A coupled wave-ocean system}.  \$316K.

\item Partner Investigator for the  Australian Research Council (2016-2020) funded project: {\it An Australian Consortium for Eddy-Resolving Ocean-Sea Ice Modelling}, AU\$599K.

\item US Department of Energy (15Aug2014--14Aug2017): {\it Three-dimensional structure of the Southern Ocean overturning circulation},  \$624K.

\item US National Science Foundation (01Sep2014--31Aug2020): {\it Southern Ocean Carbon and Climate Observations and Modeling (SOCCOM)}, \$20.9M

\item NASA (26Jun2014--25 Jun2017): {\it The role of mesoscale eddies in cross-frontal transport and subduction of nutrients and carbon in the Southern Ocean}, \$715K.

\item NOAA (01Sept2013--31Aug2016): {\it Signature of the Atlantic meridional overturning circulation in the North Atlantic dynamic sea level}, \$393K.

\item US Department of Energy (15Sep2011--14Sep2015): {\it Mode and intermediate waters in Earth System Models}, \$519K.

\item Partner Investigator for the Australian Research Council (2011-2018) Centre of Excellence for Climate System Science, AU\$21.4M.
  
\item NOAA Climate Program Office and US National Science Foundation (2010--2015): {\it Climate Processes Team on representing internal-wave driven mixing in global ocean models}.

\item NOAA Climate Program Office and US National Science Foundation (2003--2008): {\it Climate Processes Team on ocean eddy mixed layer interactions}.

\item NOAA Climate Program Office and US National Science Foundation (2003--2008): {\it Climate Processes Team on gravity current entrainment}.

\end{itemize}



\noindent\rule{\textwidth}{1pt}
\vspace{-1cm}
\section*{\sc  \color{Maroon}  convener/organizer of workshops \& meetings}
\vspace{-.3cm}

\begin{itemize}[leftmargin=*]

\item Oct 2021: Scientific advisory committee for the WCRP/CLIVAR workshop: Future Directions in Basin and Global High-resolution Ocean Modelling, GEOMAR, Kiel, Germany (virtual).

\item Mar 2019: Scientific advisory committee for the WCRP/CLIVAR workshop: Sources and sinks of ocean mesoscale eddy energy, Florida State University, Tallahassee, Florida, USA. 

\item Feb 2018: Co-convener for the Town Hall: Process understanding and standardized assessment towards the eddying realm. {\sc American Geophysical Union Ocean Sciences Conference}, Portland, Oregon, USA.

\item Feb 2018: Co-convener for the session: Modeling the Climate System at High Resolution, {\sc American Geophysical Union Ocean Sciences Conference}, Portland, Oregon, USA.

\item Sep 2016: Science Organizing Committee and Executive Planning Team for {\sc CLIVAR Open Science Conference}, Qingdao, China.

\item Apr 2014: {\sc Physical and biogeochemical ocean modelling: development, assessment, and applications}, Session at the European Geosciences Union General Assembly, Vienna, Austria.

\item Feb 2014: {\sc Physical and biogeochemical ocean modeling: development, assessment and applications}, Session at the Ocean Sciences meeting, Honolulu, Hawaii.

\item Apr 2013: {\sc Physical and biogeochemical ocean modelling: development, assessment, and applications}, Session at the European Geosciences Union General Assembly, Vienna, Austria.

\item Feb 2013: {\sc CLIVAR WGOMD/SOP Workshop on Sea-Level Rise, Ocean/Ice-Shelf Interactions, and Ice Sheets}, Hobart, Australia.  

\item Apr 2012: {\sc Physical and biogeochemical ocean modelling: development, assessment, and applications}, Session at the European Geosciences Union General Assembly, Vienna, Austria.

\item Oct 2011: {\sc Ocean Circulation and Ventilation}, Session at the WCRP Open Science Conference, Denver, USA. 

\item Apr 2011: {\sc Physical and biogeochemical ocean modelling: development, assessment, and applications}, Session at the European Geosciences Union General Assembly, Vienna, Austria.

\item Oct 2009: {\sc Workshop on Ocean Climate Modeling},
  GFDL/Princeton, USA.

 \item Apr 2009: {\sc CLIVAR Workshop on Ocean Mesoscale Eddies: Observations, Simulations, and Parameterizations}, Exeter, UK.

\item Aug 2007: {\sc CLIVAR Workshop on Numerical Methods in Ocean Modelling}, Bergen, Norway.

\item Nov 2005: {\sc CLIVAR Workshop on Modelling the Southern Ocean}, Hobart, Australia.

\item Jun 2004: {\sc CLIVAR Workshop on Evaluating the Ocean
  Component of IPCC Models}, Princeton, USA.

\item Aug 2002: {\sc Workshop on Z-coordinate Ocean Modeling}, Massachusetts Institute of Technology, USA.

\item Nov 1999: {\sc Meeting of Z-coordinate Ocean Modeling at GFDL, LANL, MIT, and NCAR}, Princeton, USA.

\item Jul 1999: {\sc Ocean/Atmosphere Variability and
  Predictability}, Session at the International Union of Geodesy and Geophysics, Birmingham, UK.

\end{itemize}




\noindent\rule{\textwidth}{1pt}
\vspace{-1cm}
\section*{\sc  \color{Maroon}  invited pedagogical lectures and special topics courses}
\vspace{-.3cm}

\begin{itemize}[leftmargin=*]

\item Oct 2022: {\sc Fundamental equations and diagnostics for MOM6}. Lecture given as part of a 2-day tutorial on  the Modular Ocean Model (MOM6), Princeton, USA.  

\item April/May 2019: {\sc Fundamentals of ocean models and the analysis of ocean simulations}. 15 lectures (45 minutes each) on ocean model fundamentals and analysis methods given as part of the {\bf Advanced Ocean Modelling Summer School}, Tasmania, Australia. 

\item Jan 2019: {\sc Ocean circulation
as a problem in mathematical \& computational physics:  a historical and contemporary  perspective}. Public lecture given as part of the Australian Mathematics Science Institute (AMSI) Summer School at the University of New South Wales, Sydney, Australia. 

\item Jul 2016: {\sc Ocean Modelling and sea level analysis}: three lectures (two hours each) at the International Centre for Theoretical Physics / Indian Institute for Tropical Meteorology: {\sc Advanced School on Earth System Modelling}, Pune, India

\item Aug 2013: {\sc Ocean models and ocean modeling: lectures on the fundamentals and practices}: Five lectures (two hours each) at the International Centre for Theoretical Physics School: {\sc Fundamentals of Ocean Climate Modeling at Global and Regional Scales}, Hyderabad, India

\item Mar 2009: {\sc Physical Processes Setting the Ocean's Water Masses}: four lectures (two hours each) at the Universit\'e Catholique de Louvain, Belgium

\item Nov 2007: {\sc Ocean Model Fundamentals}: 10 lectures (two hours each) at the University of Tasmania, Australia 

\item Aug 2006: {\sc Ocean Model Fundamentals}: two lectures (one hour each) at the NSF summer school, {\sc Modern Mathematical Methods in
Physical Oceanography}, Breckenridge, USA

\item Oct 2004: {\sc Ocean Model Fundamentals}: 10 lectures (two hours each) at the {\sc Indian Intensive School on Large-Scale Ocean Modelling}, Bangalore, India

\item Sep 2004: {\sc Ocean Model Fundamentals}: three lectures (two hours each) at the {\sc Global Ocean Data Assimilation Experiment  Summer School}, La Londe Les Maures, France

\item May 2003: {\sc Ocean Climate Modeling at NOAA-GFDL}: two lectures (one hour each) for a workshop on ocean modeling, Hobart, Australia

\item May 2002: {\sc Ocean Climate Modeling with MOM4}: three lectures (one hour each) for a workshop on ocean modeling, Kiel, Germany

\item Jan 2001: {\sc Ocean Dynamics and Modeling}: three lectures (two hours each) at La Escuela de Verano de Universidad de Concepci\'on, Chile

\item Mar 1999: {\sc Ocean and Climate Modeling}: two lectures (90 minutes each) at {\sc Conference on Global Climate}, Barcelona,
Spain

\end{itemize}


\noindent\rule{\textwidth}{1pt}
\vspace{-1cm}
\section*{\sc  \color{Maroon} pedagogical media outreach}
\vspace{-.3cm}

\begin{itemize}[leftmargin=*]

\item 2016: \href{https://www.youtube.com/watch?v=gaFjlZxM7S4&feature=youtu.be}{Animation of  the ocean's role in El Ni\~{n}o} 

\item 2015:  \href{https://www.youtube.com/watch?v=8VMSF28J9H4&list=PL9poquLHLLO91iC_6pujn6bsMCvMyJ3xU}{Animation  of  Southern Ocean circulation} 

\item 2011:  \href{https://vimeo.com/27076776}{Animation  of  ocean surface temperatures from an eddying climate model} 

\end{itemize}



\section*{\sc \color{Maroon} Presentations since 2010}
\vspace{-.3cm}

\begin{itemize}[leftmargin=*]

\item April 2024: {\sc Modeling the ocean with MOM6: A primer on vertical Lagrangian remapping and the analysis of finite volume ocean models}, Virtual seminar given at University of Rhode Island Graduate School of Oceanography. 

\item Feb 2022: {\sc A mathematical formalism for circulation in water mass configuration space}, Virtual AGU Ocean Sciences Meeting.

\item Feb 2022: {\sc meditation for scientists} (with Jonathan Lilly): {\sc American Geophysical Union Ocean Sciences Conference}, Honolulu, Hawaii (virtual).

\item May 2020: {\sc The importance of refined model resolution for understanding and projecting global, regional, and coastal sea level}, NASA GISS Virtual Sea Level Seminar Series. 

\item Feb 2020: {\sc meditation for scientists} (with Jonathan Lilly): {\sc American Geophysical Union Ocean Sciences Conference}, San Diego, CA. 

\item Jan-April 2020: {\sc Vertical Lagrangian-remapping, generalized vertical coordinates, and spurious diapycnal mixing in ocean
  models}: COMMODORE meeting, Hamburg, Germany; DRAKKAR meeting, Grenoble, France; AGU Ocean Sciences, San Diego, USA; CESM Ocean Model Working Group.  

\item Oct 2019: {\sc Water mass transformation (WMT) analysis
  and tracer budgets with generalized vertical coordinates and vertical Lagrangian-remapping}, annual meeting of the Transient tracer-based Investigation of Circulation and Thermal Ocean Change (TICTOC) Project, Exeter, UK.

\item Feb 2019: {\sc water mass transformation analysis in ocean models: some thoughts and questions}, workshop on Water mass transformation for ocean physics and biogeochemistry, University of New South Wales, Sydney, Australia.  

\item Jan 2019: {\sc A historical survey of neutral diffusion methods and comments on current research}, presented during the celebration of Peter Gent's career, NCAR, Colorado, USA.  

\item May 2018: {\sc Understanding and projecting global, regional, and coastal sea level: Reasons to include coastal ocean processes in global models}: Consortium for Ocean-Sea Ice Modelling in Australia (COSIMA) Annual Meeting, Australian National University, Canberra, Australia \& University of New South Wales, Sydney, Australia; ISSI workshop on understanding the relationship between coastal sea level and large-scale ocean circulation, Bern, Switzerland. 

\item Feb 2018: {\sc Subsurface Warming of Antarctic Coastal Waters: a Role for Both Winds and Freshening}: {\sc American Geophysical Union Ocean Sciences Conference}, Portland, Oregon, USA.

\item Dec 2017: {\sc Localized Rapid Warming of West Antarctic Subsurface Waters by Remote Winds}: American Geophysical Union Fall Meeting, New Orleans, Louisiana, USA. 

\item Nov 2017: {\sc Physical mechanisms of sea level variations in a changing climate}: International CLIVAR Scientific Steering Group meeting, Indian Institute of Tropical Meteorology, Pune, India.


\item Jul 2017: {\sc Localized rapid warming of West Antarctic
    subsurface waters by remote winds}: WCRP Conference on Regional
  Sea-level Changes and Coastal Impacts, Columbia University, New York City, USA. 

\item May 2017: {\sc Localized rapid warming of West Antarctic
    subsurface waters by remote winds}: {\it RRS JC Ross} research cruise
  JR16005 to Orkney Passage, Southern Ocean.

\item Jan 2017: {\sc The ocean mesoscale: observations, theory, and
    modeling}: Banff International Research Station (BIRS) workshop:
  {\it Transport in unsteady flows: From deterministic structures to
    stochastic models and back again}, Banff, Canada.

\item July 2016: {\sc Elements of sea level in a changing climate}:
  Indian Institute of Tropical Meteorology, Pune, India.

\item July 2016: {\sc Ocean modelling: an introduction for
    mathematical physicists}: Department of Mathematics, Savitribai
  Phule Pune University, Pune, India.

\item May 2016: {\sc Elements of sea level in a changing climate}:
  University of New South Wales, Sydney, Australia \& Australian
  National University, Canberra, Australia.

\item Jan 2016: {\sc Elements of sea level in a changing climate}:
  Louisiana State University Chemical Engineering Department, Baton
  Rouge, Louisiana, USA.

\item Oct 2015: {\sc Impacts on ocean heat from the mesoscale}:
  Lamont-Doherty Earth Observatory / Columbia University, USA.

\item Oct 2015: {\sc Impacts on ocean heat from the mesoscale}: Stony
  Brook Marine Sciences, Stony Brook, USA.

\item Oct 2014: {\sc Impacts on ocean heat from the mesoscale}:
  Meeting on ocean heat uptake at National Oceanography Centre,
  Southampton, UK.

\item Jun 2014: {\sc Impacts on ocean heat from the mesoscale}:
  University of Stockholm, Sweden.

\item Apr 2014: {\sc Problems and prospects with ocean mesoscale
    eddying climate models}: Nansen Medal lecture at the European
  Geosciences Union annual meeting, Vienna, Austria.

\item Apr 2014: {\sc Problems and prospects with ocean mesoscale
    eddying climate models}: lecture given at a CLIVAR workshop on
  eddying ocean climate models, Kiel, Germany.

\item Sep 2013: {\sc Problems and prospects of model comparisons: an
    ocean process perspective}: lecture given at a symposium
  celebrating the 80th birthday of Gerold Siedler, Kiel, Germany.

\item Feb 2013: {\sc Sea level in a suite of forced global ocean-ice
    simulations}: CLIVAR workshop on Sea-Level Rise, Ocean/Ice-Shelf
  Interactions, and Ice Sheets, Hobart, Australia

\item Jan 2013: {\sc Ocean model numerics and physics: challenges for
    mesoscale eddying global climate simulations}: 10th annual meeting
  of the Drakkar Ocean Modelling Consortia, Grenoble, France

\item Sep 2012: {\sc Sea level in ocean climate models: fundamentals
    and practices}: University of Tasmania, Hobart, Australia

\item Sep 2012: {\sc Ocean modelling with MOM and its relation to
    Australian ocean climate science}: Second meeting of Consortia for
  Ocean Modelling in Australia, Hobart, Australia

\item Feb 2012: {\sc Ocean modelling with MOM and its relation to
    Australian ocean climate science}: First meeting of Consortia for
  Ocean Modelling in Australia, Hobart, Australia

\item Mar 2011: {\sc Dynamic sea level, static sea level, and the
    non-Boussinesq steric effect}: Australia National University,
  Canberra, Australia

\item Nov 2010:  {\sc Ocean Climate Modeling at GFDL}: Scientific
  Workshop for the Centre for Australian Weather and Climate Research,
  Hobart, Australia

\item Sep 2010:  {\sc Sensitivity of Atlantic ocean variability to
    ocean physics and vertical coordinate}: CLIVAR WGOMD/GSOP Workshop
  on Decadal Variability, Predictability, and Predictions:
  Understanding the Role of the Ocean. Boulder USA 

%\item Apr 2008:  {\sc Physical Problems in Simulating the Ocean Climate System}: presentation given during a workshop on Oceans and Climate at Yale University 

%\item Mar 2008:  {\sc Physical Problems in Simulating the Ocean Climate System}: presentation given during a special session on  Climate Physics at the American Physical Society's March Meeting of Condensed Matter Physics 


\end{itemize}


\noindent\rule{\textwidth}{1pt}
\vspace{-1cm}
%\input{cv_smg_pubs_6}
\input{Griffies_cv_pubs}



\end{document}
