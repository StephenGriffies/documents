

\noindent\rule{\textwidth}{1pt}
\vspace{-1cm}
\section*{\sc  \color{Maroon}  convener/organizer of workshops \& meetings}
\vspace{-.3cm}

\begin{itemize}[leftmargin=*]

\item Oct 2021: Scientific advisory committee for the WCRP/CLIVAR workshop: Future Directions in Basin and Global High-resolution Ocean Modelling, GEOMAR, Kiel, Germany (virtual).

\item Mar 2019: Scientific advisory committee for the WCRP/CLIVAR workshop: Sources and sinks of ocean mesoscale eddy energy, Florida State University, Tallahassee, Florida, USA. 

\item Feb 2018: Co-convener for the Town Hall: Process understanding and standardized assessment towards the eddying realm. {\sc American Geophysical Union Ocean Sciences Conference}, Portland, Oregon, USA.

\item Feb 2018: Co-convener for the session: Modeling the Climate System at High Resolution, {\sc American Geophysical Union Ocean Sciences Conference}, Portland, Oregon, USA.

\item Sep 2016: Science Organizing Committee and Executive Planning Team for {\sc CLIVAR Open Science Conference}, Qingdao, China.

\item Apr 2014: {\sc Physical and biogeochemical ocean modelling: development, assessment, and applications}, Session at the European Geosciences Union General Assembly, Vienna, Austria.

\item Feb 2014: {\sc Physical and biogeochemical ocean modeling: development, assessment and applications}, Session at the Ocean Sciences meeting, Honolulu, Hawaii.

\item Apr 2013: {\sc Physical and biogeochemical ocean modelling: development, assessment, and applications}, Session at the European Geosciences Union General Assembly, Vienna, Austria.

\item Feb 2013: {\sc CLIVAR WGOMD/SOP Workshop on Sea-Level Rise, Ocean/Ice-Shelf Interactions, and Ice Sheets}, Hobart, Australia.  

\item Apr 2012: {\sc Physical and biogeochemical ocean modelling: development, assessment, and applications}, Session at the European Geosciences Union General Assembly, Vienna, Austria.

\item Oct 2011: {\sc Ocean Circulation and Ventilation}, Session at the WCRP Open Science Conference, Denver, USA. 

\item Apr 2011: {\sc Physical and biogeochemical ocean modelling: development, assessment, and applications}, Session at the European Geosciences Union General Assembly, Vienna, Austria.

\item Oct 2009: {\sc Workshop on Ocean Climate Modeling},
  GFDL/Princeton, USA.

 \item Apr 2009: {\sc CLIVAR Workshop on Ocean Mesoscale Eddies: Observations, Simulations, and Parameterizations}, Exeter, UK.

\item Aug 2007: {\sc CLIVAR Workshop on Numerical Methods in Ocean Modelling}, Bergen, Norway.

\item Nov 2005: {\sc CLIVAR Workshop on Modelling the Southern Ocean}, Hobart, Australia.

\item Jun 2004: {\sc CLIVAR Workshop on Evaluating the Ocean
  Component of IPCC Models}, Princeton, USA.

\item Aug 2002: {\sc Workshop on Z-coordinate Ocean Modeling}, Massachusetts Institute of Technology, USA.

\item Nov 1999: {\sc Meeting of Z-coordinate Ocean Modeling at GFDL, LANL, MIT, and NCAR}, Princeton, USA.

\item Jul 1999: {\sc Ocean/Atmosphere Variability and
  Predictability}, Session at the International Union of Geodesy and Geophysics, Birmingham, UK.

\end{itemize}




\noindent\rule{\textwidth}{1pt}
\vspace{-1cm}
\section*{\sc  \color{Maroon}  invited pedagogical lectures and special topics courses}
\vspace{-.3cm}

\begin{itemize}[leftmargin=*]

\item Oct 2022: {\sc Fundamental equations and diagnostics for MOM6}. Lecture given as part of a 2-day tutorial on  the Modular Ocean Model (MOM6), Princeton, USA.  

\item April/May 2019: {\sc Fundamentals of ocean models and the analysis of ocean simulations}. 15 lectures (45 minutes each) on ocean model fundamentals and analysis methods given as part of the {\bf Advanced Ocean Modelling Summer School}, Tasmania, Australia. 

\item Jan 2019: {\sc Ocean circulation
as a problem in mathematical \& computational physics:  a historical and contemporary  perspective}. Public lecture given as part of the Australian Mathematics Science Institute (AMSI) Summer School at the University of New South Wales, Sydney, Australia. 

\item Jul 2016: {\sc Ocean Modelling and sea level analysis}: three lectures (two hours each) at the International Centre for Theoretical Physics / Indian Institute for Tropical Meteorology: {\sc Advanced School on Earth System Modelling}, Pune, India

\item Aug 2013: {\sc Ocean models and ocean modeling: lectures on the fundamentals and practices}: Five lectures (two hours each) at the International Centre for Theoretical Physics School: {\sc Fundamentals of Ocean Climate Modeling at Global and Regional Scales}, Hyderabad, India

\item Mar 2009: {\sc Physical Processes Setting the Ocean's Water Masses}: four lectures (two hours each) at the Universit\'e Catholique de Louvain, Belgium

\item Nov 2007: {\sc Ocean Model Fundamentals}: 10 lectures (two hours each) at the University of Tasmania, Australia 

\item Aug 2006: {\sc Ocean Model Fundamentals}: two lectures (one hour each) at the NSF summer school, {\sc Modern Mathematical Methods in
Physical Oceanography}, Breckenridge, USA

\item Oct 2004: {\sc Ocean Model Fundamentals}: 10 lectures (two hours each) at the {\sc Indian Intensive School on Large-Scale Ocean Modelling}, Bangalore, India

\item Sep 2004: {\sc Ocean Model Fundamentals}: three lectures (two hours each) at the {\sc Global Ocean Data Assimilation Experiment  Summer School}, La Londe Les Maures, France

\item May 2003: {\sc Ocean Climate Modeling at NOAA-GFDL}: two lectures (one hour each) for a workshop on ocean modeling, Hobart, Australia

\item May 2002: {\sc Ocean Climate Modeling with MOM4}: three lectures (one hour each) for a workshop on ocean modeling, Kiel, Germany

\item Jan 2001: {\sc Ocean Dynamics and Modeling}: three lectures (two hours each) at La Escuela de Verano de Universidad de Concepci\'on, Chile

\item Mar 1999: {\sc Ocean and Climate Modeling}: two lectures (90 minutes each) at {\sc Conference on Global Climate}, Barcelona,
Spain

\end{itemize}


%\noindent\rule{\textwidth}{1pt}
%\vspace{-1cm}
%\section*{\sc  \color{Maroon} pedagogical media outreach}
%\vspace{-.3cm}

%\begin{itemize}[leftmargin=*]

%\item 2016: \href{https://www.youtube.com/watch?v=gaFjlZxM7S4&feature=youtu.be}{Animation of  the ocean's role in El Ni\~{n}o} 

%\item 2015:  \href{https://www.youtube.com/watch?v=8VMSF28J9H4&list=PL9poquLHLLO91iC_6pujn6bsMCvMyJ3xU}{Animation  of  Southern Ocean circulation} 

%\item 2011:  \href{https://vimeo.com/27076776}{Animation  of  ocean surface temperatures from an eddying climate model} 


%\end{itemize}



\section*{\sc \color{Maroon} Presentations since 2010}
\vspace{-.3cm}

\begin{itemize}[leftmargin=*]

\item Mar 2025: {\sc Climate model thermal equilibration in an ocean mesoscale dominant regime}, virtual talk to the Harvard University Earth and Planetary Sciences Department.

\item Dec 2024: {\sc MOM6 and the CM4X Climate Model Hierarchy}, talk given at the annual meeting of the American Geophysical Union, Washington, DC, USA. 

\item Oct 2024: {\sc MOM6 and the CM4X Climate Model Hierarchy}, talk given at the Center for Ocean/Atmosphere Science, Courant Institute of Mathematical Sciences, New York University, USA. 

\item Sep 2024: {\sc The Long and Winding Road to OMIP}, talk given at the 50th anniversary of the Ocean Section of NCAR, National Center of Atmospheric Research, Boulder, USA. 

\item Sep 2024: {\sc MOM6 and the CM4X Climate Model Hierarchy}, talk given at the COMMODORE workshop on numerical methods in ocean modeling, National Center of Atmospheric Research, Boulder, USA. 

\item Aug 2024: {\sc MOM6 and the CM4X Climate Model Hierarchy}, talk given at the annual Eddy Energy and Transport Climate Process Team meeting, Brown University, USA. 

\item June 2024: {\sc Modeling the ocean with MOM6: A primer on vertical Lagrangian remapping and the analysis of finite volume ocean models}, virtual seminar given at Alfred Wegener Institute, Germany, in honor of the retirement of {R\"{u}diger} Gerdes. 

\item April 2024: {\sc Modeling the ocean with MOM6: A primer on vertical Lagrangian remapping and the analysis of finite volume ocean models}, virtual seminar given at University of Rhode Island Graduate School of Oceanography. 

\item Feb 2022: {\sc A mathematical formalism for circulation in water mass configuration space}, virtual presentation at AGU Ocean Sciences Meeting.

\item Feb 2022: {\sc meditation for scientists} (with Jonathan Lilly): {\sc American Geophysical Union Ocean Sciences Conference}, Honolulu, Hawaii (virtual).

\item May 2020: {\sc The importance of refined model resolution for understanding and projecting global, regional, and coastal sea level}, NASA GISS virtual Sea Level Seminar Series. 

\item Feb 2020: {\sc meditation for scientists} (with Jonathan Lilly): {\sc American Geophysical Union Ocean Sciences Conference}, San Diego, CA. 

\item Jan-April 2020: {\sc Vertical Lagrangian-remapping, generalized vertical coordinates, and spurious diapycnal mixing in ocean
  models}: COMMODORE meeting, Hamburg, Germany; DRAKKAR meeting, Grenoble, France; AGU Ocean Sciences, San Diego, USA; CESM Ocean Model Working Group.  

\item Oct 2019: {\sc Water mass transformation (WMT) analysis
  and tracer budgets with generalized vertical coordinates and vertical Lagrangian-remapping}, annual meeting of the Transient tracer-based Investigation of Circulation and Thermal Ocean Change (TICTOC) Project, Exeter, UK.

\item Feb 2019: {\sc water mass transformation analysis in ocean models: some thoughts and questions}, workshop on Water mass transformation for ocean physics and biogeochemistry, University of New South Wales, Sydney, Australia.  

\item Jan 2019: {\sc A historical survey of neutral diffusion methods and comments on current research}, presented during the celebration of Peter Gent's career, NCAR, Colorado, USA.  

\item May 2018: {\sc Understanding and projecting global, regional, and coastal sea level: Reasons to include coastal ocean processes in global models}: Consortium for Ocean-Sea Ice Modelling in Australia (COSIMA) Annual Meeting, Australian National University, Canberra, Australia \& University of New South Wales, Sydney, Australia; ISSI workshop on understanding the relationship between coastal sea level and large-scale ocean circulation, Bern, Switzerland. 

\item Feb 2018: {\sc Subsurface Warming of Antarctic Coastal Waters: a Role for Both Winds and Freshening}: {\sc American Geophysical Union Ocean Sciences Conference}, Portland, Oregon, USA.

\item Dec 2017: {\sc Localized Rapid Warming of West Antarctic Subsurface Waters by Remote Winds}: American Geophysical Union Fall Meeting, New Orleans, Louisiana, USA. 

\item Nov 2017: {\sc Physical mechanisms of sea level variations in a changing climate}: International CLIVAR Scientific Steering Group meeting, Indian Institute of Tropical Meteorology, Pune, India.


\item Jul 2017: {\sc Localized rapid warming of West Antarctic
    subsurface waters by remote winds}: WCRP Conference on Regional
  Sea-level Changes and Coastal Impacts, Columbia University, New York City, USA. 

\item May 2017: {\sc Localized rapid warming of West Antarctic
    subsurface waters by remote winds}: {\it RRS JC Ross} research cruise
  JR16005 to Orkney Passage, Southern Ocean.

\item Jan 2017: {\sc The ocean mesoscale: observations, theory, and
    modeling}: Banff International Research Station (BIRS) workshop:
  {\it Transport in unsteady flows: From deterministic structures to
    stochastic models and back again}, Banff, Canada.

\item July 2016: {\sc Elements of sea level in a changing climate}:
  Indian Institute of Tropical Meteorology, Pune, India.

\item July 2016: {\sc Ocean modelling: an introduction for
    mathematical physicists}: Department of Mathematics, Savitribai
  Phule Pune University, Pune, India.

\item May 2016: {\sc Elements of sea level in a changing climate}:
  University of New South Wales, Sydney, Australia \& Australian
  National University, Canberra, Australia.

\item Jan 2016: {\sc Elements of sea level in a changing climate}:
  Louisiana State University Chemical Engineering Department, Baton
  Rouge, Louisiana, USA.

\item Oct 2015: {\sc Impacts on ocean heat from the mesoscale}:
  Lamont-Doherty Earth Observatory / Columbia University, USA.

\item Oct 2015: {\sc Impacts on ocean heat from the mesoscale}: Stony
  Brook Marine Sciences, Stony Brook, USA.

\item Oct 2014: {\sc Impacts on ocean heat from the mesoscale}:
  Meeting on ocean heat uptake at the National Oceanography Centre,
  Southampton, UK.

\item Jun 2014: {\sc Impacts on ocean heat from the mesoscale}:
  University of Stockholm, Sweden.

\item Apr 2014: {\sc Problems and prospects with ocean mesoscale
    eddying climate models}: Nansen Medal lecture at the European
  Geosciences Union annual meeting, Vienna, Austria.

\item Apr 2014: {\sc Problems and prospects with ocean mesoscale
    eddying climate models}: lecture given at a CLIVAR workshop on
  eddying ocean climate models, Kiel, Germany.

\item Sep 2013: {\sc Problems and prospects of model comparisons: an
    ocean process perspective}: lecture given at a symposium
  celebrating the 80th birthday of Gerold Siedler, Kiel, Germany.

\item Feb 2013: {\sc Sea level in a suite of forced global ocean-ice
    simulations}: CLIVAR workshop on Sea-Level Rise, Ocean/Ice-Shelf
  Interactions, and Ice Sheets, Hobart, Australia

\item Jan 2013: {\sc Ocean model numerics and physics: challenges for
    mesoscale eddying global climate simulations}: 10th annual meeting
  of the Drakkar Ocean Modelling Consortia, Grenoble, France

\item Sep 2012: {\sc Sea level in ocean climate models: fundamentals
    and practices}: University of Tasmania, Hobart, Australia

\item Sep 2012: {\sc Ocean modelling with MOM and its relation to
    Australian Ocean Climate Science}: Second meeting of Consortia for
  Ocean Modelling in Australia, Hobart, Australia

\item Feb 2012: {\sc Ocean modelling with MOM and its relation to
    Australian Ocean Climate Science}: First meeting of Consortia for
  Ocean Modelling in Australia, Hobart, Australia

\item Mar 2011: {\sc Dynamic sea level, static sea level, and the
    non-Boussinesq steric effect}: Australia National University,
  Canberra, Australia

\item Nov 2010:  {\sc Ocean Climate Modeling at GFDL}: Scientific
  Workshop for the Centre for Australian Weather and Climate Research,
  Hobart, Australia

\item Sep 2010:  {\sc Sensitivity of Atlantic ocean variability to
    ocean physics and vertical coordinate}: CLIVAR WGOMD/GSOP Workshop
  on Decadal Variability, Predictability, and Predictions:
  Understanding the Role of the Ocean. Boulder USA 

%\item Apr 2008:  {\sc Physical Problems in Simulating the Ocean Climate System}: presentation given during a workshop on Oceans and Climate at Yale University 

%\item Mar 2008:  {\sc Physical Problems in Simulating the Ocean Climate System}: presentation given during a special session on  Climate Physics at the American Physical Society's March Meeting of Condensed Matter Physics 


\end{itemize}
