%LATEX 
%\documentclass[10pt,draft]{beamer}
\documentclass[10pt]{beamer}
\usepackage{beamerthemesplit} % new 


\usepackage{bm,amsmath,amssymb}
\usepackage{mathtools,slashed,esint}

\usepackage{animate}


\usepackage{pgf}
\usepackage{wasysym}
\usepackage{xcolor}
\usepackage[english]{babel}
\usepackage[latin1]{inputenc}
\usepackage{times}
\usepackage[T1]{fontenc}
\usepackage{wasysym}
\usepackage{centernot}

\usepackage{tikz}
\usepackage{pgf}
\usepackage{xcolor}
\usepackage[english]{babel}
\usepackage[latin1]{inputenc}
\usepackage{times}
\usepackage[T1]{fontenc}

%\usepackage{graphics}
\usepackage{graphicx}
\usepackage{media9}
\usepackage{natbib}
\usepackage{color}

%\usepackage{../../../../../Documents/TEX/styles/mystyle_beamer}

\usepackage{ragged2e}
\usepackage{etoolbox}
\usepackage{lipsum}

\apptocmd{\frame}{\justifying}{}{}

\usepackage{hyperref}  % must be used last in the list of packages 

\hypersetup{colorlinks=true,linkcolor=green,citecolor=green,filecolor=green,urlcolor=green} 


\setlength\fboxsep{0pt}
\setlength\fboxrule{0.5pt}

\newcommand\myeq{\mathrel{\overset{\makebox[0pt]{\mbox{\normalfont\tiny\sffamily M constant}}}{=}}}
\newcommand\myeqq{\mathrel{\overset{\makebox[0pt]{\mbox{\normalfont\tiny\sffamily Stokes Thm}}}{=}}}

\newcommand{\notimplies}{%
  \mathrel{{\ooalign{\hidewidth$\not\phantom{=}$\hidewidth\cr$\implies$}}}}

\newcommand{\Simley}[1]{%
\begin{tikzpicture}[scale=0.11]
    \newcommand*{\SmileyRadius}{1.0}%
    \draw [fill=brown!10] (0,0) circle (\SmileyRadius)% outside circle
        %node [yshift=-0.22*\SmileyRadius cm] {\tiny #1}% uncomment this to see the smile factor
        ;  

    \pgfmathsetmacro{\eyeX}{0.5*\SmileyRadius*cos(30)}
    \pgfmathsetmacro{\eyeY}{0.5*\SmileyRadius*sin(30)}
    \draw [fill=cyan,draw=none] (\eyeX,\eyeY) circle (0.15cm);
    \draw [fill=cyan,draw=none] (-\eyeX,\eyeY) circle (0.15cm);

    \pgfmathsetmacro{\xScale}{2*\eyeX/180}
    \pgfmathsetmacro{\yScale}{1.0*\eyeY}
    \draw[color=red, domain=-\eyeX:\eyeX]   
        plot ({\x},{
            -0.1+#1*0.15 % shift the smiley as smile decreases
            -#1*1.75*\yScale*(sin((\x+\eyeX)/\xScale))-\eyeY});
\end{tikzpicture}%
}%



% beamer prologue begins here
\mode<presentation>
{                               % see user guide for choices
  \usetheme{Frankfurt}
%  \usetheme{Madrid}
%  \usecolortheme{seahorse}
%  \usecolortheme[RGB={205,173,0}]{structure} 
  \usecolortheme[RGB={0,100,180}]{structure} 
  \usefonttheme[onlymath]{serif}
  \setbeamercovered{transparent} % or whatever (possibly just delete it)
}

% drop those navigation symbols at bottom right
%\setbeamertemplate{navigation symbols}{} 
\setbeamertemplate{navigation symbols}{}
\setbeamertemplate{footline}[frame number]


\title[WMT and numerical models]{WMT analysis in ocean models: \\ some
  thoughts and questions} % (optional)

\subtitle{ {\it Watermass Transformation Meeting 2019} \\
  4-6  Feb 2019 \\ \vspace{.25cm}
  University of New South Wales \\
 Sydney, Australia  \\ Traditional lands of the Bedegal people of the Eora Nation
%\vspace{.25cm}
%\fbox{\includegraphics[width=.3\textwidth]{../../../../../Griffies/lectures/India_2013/griffies/figs/cm2p6.png}}
}

\author[{\sc Stephen.M.Griffies@gmail.com}] % short form for footer, use if many authors
{ {\sc Stephen.Griffies@noaa.gov  \\ SMG@Princeton.edu \\  Stephen.M.Griffies@gmail.com 
} 
}
% - Use the \inst{?} command only if the authors have different 
%   affiliation. 



\institute[NOAA/GFDL + Princeton University]{NOAA/GFDL + Princeton University} 

%\date{\today}
\date{}


%\logo{\includegraphics[height=5mm]{../../../../../Documents/images/PrincetonLogo.png}} 
\logo{\includegraphics[height=5mm]{../../../../Documents/images/noaa-logo.png}} 

\subject{Talks} % for "PDF information catalog", can be left out. 

% this shows the outline at the beginning of every section,
% highlighting the current section
 \AtBeginSection[]
 {
   \begin{frame}
     \frametitle{Outline}
 \tableofcontents[ 
 currentsubsection, 
 hideothersubsections, 
 sectionstyle=show/hide, 
 subsectionstyle=show/hide, 
 ] 
   \end{frame}
 }


\setlength{\unitlength}{1mm}  % used in picture environment

\begin{document}
\footnotesize 

  
% beamer makes the titlepage from info above: author, date, title, etc.
\begin{frame}
  \titlepage
\end{frame}




\begin{frame}
  \frametitle{Goals and apologies}

\begin{itemize}

\item[$\star$] {\sc conceptual and practical}: Aspects of WMT analysis that
  impacts our ability to use the method in numerical models
  \href{https://www.annualreviews.org/doi/abs/10.1146/annurev-marine-010318-095421}{(largely
    from Groeskamp et al 2019).}

\item [$\star$]  {\sc model algorithms and budgets}: How do models evolve
  tracers?  
\begin{itemize} \footnotesize 
\item[] $\longrightarrow$ A bit on quasi-Eulerian (traditional) and
  quasi-Lagrangian (and ALE) methods.
    \end{itemize}

\item[$\star$] {\sc Take-home point}: To write rigorous and thorough
  code for 3d process-based WMT analysis requires expertise in WMT
  analysis, process physics and BGC, and numerical algorithms.
    \begin{itemize} \footnotesize 
       \item[] $\longrightarrow$ It is for us to determine if the investment has sufficient payoff. 
    \end{itemize}


  \item[$\star$] {\sc Apologies} for the heaps of maths. \\
    ``Everything should be made as simple as possible, but not
    simpler.''  A. Einstein.

\end{itemize}


\end{frame}




% beamer makes the ToC from sections and subsections below...
\begin{frame}
  \frametitle{Outline}
  \tableofcontents
\end{frame}




\section{The WMT elevator speech}


\begin{frame}
  \frametitle{Basic mathematical expressions
\small \href{https://www.annualreviews.org/doi/abs/10.1146/annurev-marine-010318-095421}{(Groeskamp et al 2019)}
}

{\sc Measure seawater transport across iso-surfaces of a scalar
  property.}  \scriptsize Assume smooth $\lambda$ surfaces
$\Longrightarrow$ average over microstructure as per
\href{https://doi.org/10.1175/1520-0485(1993)023<2254:MFADS>2.0.CO;2}{De
  Szoeke and Bennett (1993).}  \footnotesize


\begin{center}
\vspace{-.1cm}
{\includegraphics[width=0.4\textwidth]{./figs/Groeskamp_etal_fig3.pdf}}
\end{center}
\vspace{-.5cm}


\begin{equation}
G(\lambda) = \iint\limits_{\lambda} \rho \, u^{\mbox{\tiny dia}} \, \mathrm{d}A = 
\iint\limits_{\lambda} \frac{\rho \, \dot{\lambda} \, \mathrm{d}A}{|\nabla \lambda|}
\label{eq:wmt-defined-math}
\end{equation}
 Fundamental theorem of calculus renders the more useful form 
\begin{equation}
G(\lambda) = 
\frac{\partial}{\partial \lambda}
\iiint\limits_{\lambda' \le \lambda}
\rho \, \dot{\lambda}' \, \mathrm{d}V.
\label{eq:wmt-identity-math}
\end{equation}
Discretized form for numerical calculations 
\begin{equation}
G(\lambda) 
\approx 
\underbrace{
  \frac{1}{\Delta \lambda} \sum_{n=1}^{N} \sum_{i,j,k}
  \Pi(\lambda_{n},\lambda,\Delta \lambda)
  \, (\rho \, \dot{\lambda})_{i,j,k} \, V_{i,j,k}.
}_{\mbox{\tiny Box-car binning of the material change $\rho \dot{\lambda}$  into $N$-bins}}
\label{eq:wmt-discrete-math}
\end{equation}


\end{frame}


\begin{frame}
  \frametitle{Tutorial on dia-surface transport \href{https://press.princeton.edu/titles/7797.html}{\small (Section 6.7 of Griffies 2004)}}

\vspace{-.2cm} 
\begin{center}
{\includegraphics[width=0.6\textwidth]{./figs/dia_surface_schematic.pdf}}
\end{center}

\scriptsize  

In
\href{https://www.annualreviews.org/doi/abs/10.1146/annurev-marine-010318-095421}{Groeskamp
  et al (2019),} we define the dia-surface velocity component,
$u^{\mbox{\tiny dia}}$, according to the volume of seawater
transported across a moving surface
\begin{align}
 &{\cal T} \equiv u^{\mbox{\tiny dia}}  \, \mathrm{d}S  \equiv \hat{\bm n} \cdot ({\bm v} - {\bm v}^{(\lambda)})  \, \mathrm{d}S
 \Longrightarrow  \mbox{can have non-zero transport even if $\hat{\bm n} \cdot {\bm v} =0$.}
  \\
&\hat{\bm n} = \frac{\nabla \lambda}{|\nabla \lambda|}    = \; \; \mbox{surface outward normal direction} 
 \\
&{\bm v}  = \mbox{velocity of fluid element}   \qquad \mbox{and} \qquad 
   \frac{\partial \lambda}{\partial t} +   {\bm v}^{(\lambda)} \cdot \nabla \lambda = 0.
\end{align}
 Following from these definitions we have 
\begin{align}
  \frac{\mathrm{D}\lambda}{\mathrm{D}t} 
  &= \frac{\partial \lambda}{\partial t} + {\bm v} \cdot \nabla \lambda 
 = \frac{\partial \lambda}{\partial t} + {\bm v}^{(\lambda)} \cdot \nabla \lambda  + ({\bm v} - {\bm v}^{(\lambda)}) \cdot \nabla \lambda 
= 0 + u^{\mbox{\tiny dia}} \, |\nabla \lambda| 
\\
 &\Longrightarrow u^{\mbox{\tiny dia}}  = \frac{1}{|\nabla \lambda|} \, \frac{\mathrm{D}\lambda}{\mathrm{D}t}  = \frac{\dot{\lambda}}{|\nabla \lambda|}
\Longrightarrow \mbox{material changes in $\lambda$ yield dia-surface transport.}
\end{align}
Note that for stably stratified $\lambda$-surfaces (as in generalized
vertical coordinate models), we define
\begin{equation}
   {\cal T} \equiv u^{\mbox{\tiny dia}}  \mathrm{d}S  
  \equiv  w^{\lambda}  \mathrm{d}A = (\partial z/ \partial \lambda) \, \dot{\lambda} 
\end{equation}


\end{frame}



\begin{frame}
  \frametitle{Notable conceptual characteristics
\small \href{https://www.annualreviews.org/doi/abs/10.1146/annurev-marine-010318-095421}{(Groeskamp et al 2019)}
}

\begin{center}
\vspace{-.1cm}
{\includegraphics[width=0.45\textwidth]{./figs/Groeskamp_etal_fig3.pdf}}
\end{center}
\vspace{-.3cm}

\begin{exampleblock}{}
\begin{itemize}

\item  Suitable kinematic framework for studying tracer dynamics. 
 \begin{itemize}
 \footnotesize 
   \item[$\star$] $\lambda$ surfaces can be non-monotonic and disconnected.
   \item[$\star$] $\lambda$ is typically density (which one?), temperature, or salinity. 
   \item[$\star$]  $\lambda$ perhaps could also age, PV, carbon, others.  
   \item[$\star$] Geographical non-locality $\Longrightarrow$ distinct from strict isopycnal analysis.
   \item[$\star$] Generalizable to $\Theta-S$ space ({D\"{o}\"{o}s},
     Zika, Groeskamp, collaborators).
   \item[$\star$] Generalizable to arbitrary 3-scalar space (Nurser,
     Zika, Griffies ongoing).
 \end{itemize}

\item Geographical non-locality means WMT phase space is {\it not}
  suitable for spatially local contact forces such as pressure and
  stress.
%\vspace{-.2cm}
\begin{itemize}
 \footnotesize 
\item[$\star$] $\Longrightarrow$ Still need Eulerian or Lagrangian
  kinematics for momentum dynamics.
\item[$\star$] $\Longrightarrow$ Perfect-fluid dynamical processes
  ($\dot{\lambda} = 0$) are invisible to WMT.
\end{itemize}

\end{itemize}
\end{exampleblock}{}

\end{frame}





\section{What we need from the models}

\begin{frame}
  \frametitle{Budget analysis}

\begin{center}
\color{red}
Models need to carefully diagnose $\rho \, \dot{\lambda} = -\nabla \cdot {\bm J}.$ 
\color{black}
\end{center}


\begin{itemize}

\item {\sc kinematic method}: Direct calculation but offers no process information 
\begin{equation}
  \underbrace{ \rho \, \dot{\lambda} }_{\mbox{\tiny material change}}
    = \underbrace{ \frac{\partial (\rho \, \lambda)}{\partial t}}_{\mbox{\tiny local tendency}}
    + \underbrace{  \nabla \cdot (\rho \, \lambda \, {\bm v}^{\dagger} ) }_{\mbox{\tiny residual mean advection}}
 \qquad  \mbox{with}  \qquad {\bm v}^{\dagger} = {\bm v} + {\bm v}^{\mbox{\tiny sgs}}. 
\end{equation}


\item {\sc process method}: Indirect calculation offering full process information
\begin{equation}
  \rho \, \dot{\lambda} = -\nabla \cdot {\bm J} = \sum (\mbox{interior processes + boundaries}).
\end{equation}

\item {\sc some numerical issues that make careful budgets tough}: 
\begin{itemize}
\footnotesize 

\item[$\star$] Models generally have changing cell thicknesses, thus
  requiring care to keep track of cell volume along with tracer
  evolution (more later).

\item[$\star$] Multi-step time stepping means ocean is incrementally
  updated; e.g., time-implicit vertical diffusion $\Longrightarrow$
  tough to diagnostically close budgets.

\item[$\star$] Numerial algorithms often preclude fine-scale
  decomposition of budgets; e.g., nonlinear numerical advection
\begin{equation}
   {\bm v}^{\dagger} = {\bm v} + {\bm v}^{\mbox{\tiny sgs}}   \; \; \;  \mbox{and yet} \; \; \; 
  \nabla \cdot ( \rho \, C \, {\bm v}^{\dagger}) \ne \nabla \cdot ( \rho \, C \, {\bm v}) + \nabla \cdot ( \rho \, C \, {\bm v}^{\mbox{\tiny sgs}}  ). 
\end{equation}
 

\end{itemize}

\end{itemize}


\end{frame}




\section{Scalar equations in ocean models}

\begin{frame}
  \frametitle{Z-coordinate scalar eqns w/ eddy-induced ${\bm v}^{*}$}

\begin{center}
\vspace{-.3cm}
{\includegraphics[width=0.4\textwidth]{./figs/z_layer.pdf}}
\end{center}
\vspace{-.5cm}

\begin{align}
 &\frac{\mathrm{D}^{\dagger}}{\mathrm{D}t} = \frac{\partial }{\partial t} + {\bm v}^{\dagger} \cdot \nabla, 
 \; \; {\bm v}^{\dagger} = {\bm v} + {\bm v}^{*},  \; \; 
 \mbox{local mass conserve eddy}  \Longrightarrow   \nabla \cdot (\rho \, {\bm v}^{*}) = 0 
\\
 &\frac{\mathrm{D}^{\dagger} ( \rho \, \delta V)}{\mathrm{D}t} = 0  \Longrightarrow  
   \frac{\partial \rho}{\partial t} + \nabla \cdot (\rho \, {\bm v}) = 0  
 \; \;  \mbox{usual continuity equation even with ${\bm v}^{*} \ne 0$}
\\
  &\rho \, \frac{\mathrm{D}^{\dagger} C }{\mathrm{D}t} = -\nabla \cdot {\bm J}  \Longrightarrow  
   \frac{\partial (\rho \, C) }{\partial t} + \nabla \cdot (\rho \, C \, {\bm v}^{\dagger} ) = - \nabla \cdot {\bm J}.
\label{eq:z-coordinate-mass-tracer-equation}
\end{align}


\vspace{-.3cm}
\begin{exampleblock}{}
\begin{itemize}
\item Some tracers, especially BGC tracers, also have sources.  Ignore here for brevity.  
\end{itemize}
\end{exampleblock}{}

\end{frame}



\begin{frame}
  \frametitle{Quasi-Eulerian scalar eqns w/ eddy-induced ${\bm v}^{*}$}

\begin{center}
\vspace{-.3cm}
{\includegraphics[width=0.45\textwidth]{./figs/gvc_layer_eddy_induced.pdf}}
\end{center}
\vspace{-.5cm}

\begin{align}
  &\frac{\partial (h \, \rho)}{\partial  t} + \nabla_{\sigma} \cdot (h \, \rho \, {\bm u}^{\dagger} ) 
 + \delta_{\sigma} (\rho \, w^{\dagger \sigma} ) 
 = 0 
 \; \; \mbox{and} \; \; \nabla_{\sigma} \cdot ( h \, \rho \, {\bm u}^{*})  + \delta_{\sigma} (\rho \, w^{*\sigma} ) = 0 
\\
  &\frac{\partial (h \, \rho \, C)}{\partial  t} + \nabla_{\sigma} \cdot (h \, \rho \, C \, {\bm u}^{\dagger} ) 
 + \delta_{\sigma} (\rho \, C \, w^{\dagger \sigma} ) 
 =
 -\left[  \nabla_{\sigma} \cdot ( h \, {\bm J}^{h}) +  \delta_{\sigma} J^{\sigma} \right] 
 \\
  &z_{\sigma} = \frac{\partial z}{\partial \sigma} 
  \qquad h = z_{\sigma} \, \mathrm{d}\sigma 
  \qquad w^{\dagger \sigma} =  \frac{\partial z}{\partial \sigma}  \frac{ \mathrm{D}^{\dagger} \sigma}{\mathrm{D}t} 
  = w^{\sigma} + {\bm v}^{*}\cdot z_{\sigma} \nabla \sigma = w^{\sigma}  + w^{*\sigma} 
 \\
 &\delta_{\sigma} = \mathrm{d}\sigma \, \frac{\partial}{\partial \sigma}
 \qquad \nabla_{\sigma} = \nabla_{z} + {\bm S} \, \partial_{z}
 \qquad {\bm S}  = \nabla_{\sigma} z = -z_{\sigma}  \nabla_{z}\sigma
 \qquad J^{\sigma}  = z_{\sigma}  \nabla \sigma  \cdot {\bm J} 
\label{eq:gvc-mass-tracer-equation}
\end{align}

\begin{exampleblock}{}
\begin{itemize}
\item Coordinate transformation of the z-coordinate equations (e.g., Griffies 2004).  
\item $w^{\sigma}$ \& $w^{*\sigma} $ are diagnosed via the continuity
  equation: a {\it kinematic} approach.
\item Quasi-Eulerian codes: MOM5, MITgcm, NEMO, ROMS.
\item Example coordinates are those that do not vanish (unless lose
  ocean to rock)  $\Longrightarrow$ isopycnal coordinates are not available. 
\end{itemize}
\end{exampleblock}{}

\end{frame}



\begin{frame}
  \frametitle{Quasi-Lagrangian scalar eqns w/ eddy-induced ${\bm u}^{\mbox{\tiny bolus}}$}

\begin{center}
\vspace{-.3cm}
{\includegraphics[width=0.5\textwidth]{./figs/gvc_layer_bolus.pdf}}
\end{center}
\vspace{-.5cm}

\begin{align}
  &\frac{\partial (h \, \rho)}{\partial  t} + \nabla_{\sigma} \cdot (h \, \rho \, {\bm u}^{\dagger} ) 
 + \delta_{\sigma} (\rho \, w^{\sigma} ) 
 = 0 
 \\
  &\frac{\partial (h \, \rho \, C)}{\partial  t} + \nabla_{\sigma} \cdot (h \, \rho \, C \, {\bm u}^{\dagger} ) 
 + \delta_{\sigma} (\rho \, C \, w^{\sigma} ) 
 =
 -\left[  \nabla_{\sigma} \cdot ( h \, {\bm J}^{h}) +  \delta_{\sigma} J^{\sigma} \right] 
 \\
  &w^{\sigma} =  \frac{\partial z}{\partial \sigma}  \frac{ \mathrm{D} \sigma}{\mathrm{D}t} 
 \qquad {\bm u}^{\dagger}  = {\bm u} + {\bm u}^{\mbox{\tiny bolus}}  = \mbox{horizontal residual mean velocity.}
\label{eq:gvc-mass-tracer-equation}
\end{align}


\vspace{-.2cm} 
\begin{exampleblock}{}
\begin{itemize}

\item $w^{\sigma}$ is specified based on dia-surface processes such as
  mixing: a {\it thermodynamic} or {\it process} approach.

\item Makes use of ${\bm u}^{\mbox{\tiny bolus}}$ to ensure advective
  transport retains layer integrated constant mass, even if not an
  adiabatic isopycnal layer.
 \begin{itemize}
 \item[$\star$] \scriptsize A choice stemming from isopycnal models,
   here chosen also for any coordinate.
 \item[$\star$] See
   \href{https://doi.org/10.1175/1520-0485(2001)031<1222:TTRMVP>2.0.CO;2}{McDougall
     and McIntosh (2001)},
   \href{https://press.princeton.edu/titles/7797.html}{Griffies
     (2004),}
   \href{https://journals.ametsoc.org/doi/full/10.1175/JPO-D-11-0102.1}{Young
     (2012)} for relation between ${\bm u}^{\mbox{\tiny bolus}}$ and
   ${\bm v}^{*}$.
\end{itemize}

\item Quasi-Lagrangian codes: HYCOM, MOM6, MPAS-O, NEMO (immature?).

%\item Can readily handle vanishing layers (wetting and drying).

\end{itemize}
\end{exampleblock}{}

\end{frame}




\begin{frame}
  \frametitle{Arbitrary Lagrangian Eulerian (ALE) method for the scalar equations \small  
\href{https://adcroft.github.io/2017/03/01/MOM6-ALE-algorithm.html}{(Based on AJ Adcroft)} }

\vspace{-.3cm} 


\begin{center}
{\includegraphics[width=0.9\textwidth]{./figs/ALE_schematic.pdf}}
\end{center}


\begin{exampleblock}{}
\begin{itemize}

\item {\sc Immiscible layer step:} Using the quasi-Lagrangian
  formulation, evolve ocean state to time $t + \Delta t$ using
  $\dot{\sigma} = 0$ (so integrated mass is constant in each layer)
  yet with the full suite of SGS parameterizations
\begin{equation*}
\frac{\partial (h \, \rho)}{\partial  t} + \nabla_{\sigma} \cdot (h \, \rho \, {\bm u}^{\dagger} ) = 0 
\qquad 
 \frac{\partial (h \, \rho \, C)}{\partial  t} + \nabla_{\sigma} \cdot (h \, \rho \, C \, {\bm u}^{\dagger} ) 
 =
 -\left[  \nabla_{\sigma} \cdot ( h \, {\bm J}^{h}) +  \delta_{\sigma} J^{\sigma} \right]. 
\label{eq:gvc-mass-tracer-equation-ALE-stepA}
\end{equation*}


\item {\sc Regridding step:} New grid with thicknesses $h^{*}$ fitting
  in the model domain.

\item {\sc Remapping step:} Take the new state and map it onto $h^*$.
  Use a very high order accurate one-dimensional scheme to minimize
  spurious mixing/unmixing.

\end{itemize}
\end{exampleblock}{}

\end{frame}



\begin{frame}
  \frametitle{Comments about ALE}

\begin{center}
{\includegraphics[width=0.9\textwidth]{./figs/ALE_schematic.pdf}}
\end{center}


\vspace{-.3cm} 
\begin{exampleblock}{}
\begin{itemize}


\item {\sc Where is dia-surface advection?}  It is part of the
  evolution of the grid cell thicknesses.  Cell interfaces move and
  carry the state.
   \begin{itemize}  \footnotesize 
   \item[$\star$] {\sc z-coordinate example:} Define $h^{*}$ according
     to fixed z-levels.  Remapping moves the state onto the fixed
     $z-$grid, a step that is the operationally same as vertical
     advection.
   \end{itemize}

 \item To diagnose the full advection operator, we need to diagnose
   the contribution from remapping so that
\begin{equation}
    \nabla \cdot (\rho \, C \, {\bm v}^{\dagger} ) = \underbrace{ \nabla_{\sigma} \cdot (\rho \, C \, {\bm u}^{\dagger} )}_{\mbox{\tiny horizontal layer advection}}
  + \; \; \mbox{remapping}.
\end{equation}

\item There is no CFL associated with vertical remapping; useful for
  fine vertical grid spacing.  But remember stability does not imply
  accuracy.

\item The vertical remapping algorithm can be used for diagnostic
  purposes to remap and bin grid cell tendencies according to
  arbitrary surfaces.

\end{itemize}
\end{exampleblock}{}

\end{frame}













\section{Ongoing issues}

\begin{frame}
  \frametitle{Spurious mixing and unmixing}

  The numerical representation of
  $\mbox{advection} = \nabla \cdot ( \rho \, C \, {\bm v}^{\dagger})$
  generally introduces spurious mixing and unmixing due to truncation
  errors
  \href{https://doi.org/10.1175/1520-0493(2000)128<0538:SDMAWA>2.0.CO;2}{(Griffies
    et al 2000 and others)}
\begin{equation}
   \nabla \cdot ( \rho \, C \, {\bm v}^{\dagger})_{\mbox{\tiny model}} = 
   \nabla \cdot ( \rho \, C \, {\bm v}^{\dagger})_{\mbox{\tiny exact}} 
  +
 \nabla \cdot ( \rho \, C \, {\bm v}^{\dagger})_{\mbox{\tiny noise}}
\end{equation}

\begin{exampleblock}{}
\begin{itemize}
\item[$\star$] Noise in numerical advection can be interpreted as an
  extra SGS term
\begin{equation}
   \frac{\partial (\rho \, \lambda)}{\partial t} + \nabla \cdot (\rho \, \lambda \, {\bm v}^{\dagger} )_{\mbox{\tiny exact}} 
 = -\nabla \cdot \left[  {\bm J} + (\rho \, \lambda \, {\bm v}^{\dagger} )_{\mbox{\tiny noise}} \right]. 
\end{equation}

\item[$\star$] Noise term is not physical.  Hence, if it is large it
  can corrupt the physical integrity of the simulation; e.g.,
  spurious watermass trends can be induced.

\item[$\star$] Noise term can become larger when refine grid spacing
  to partially resolve mesoscale eddies, which pump tracer variance to
  the grid scale.  Need to move into the very fine spacing regime
  before spurious mixing reduces:
  $0.25^{\circ} \longrightarrow 0.1^{\circ}$.

\item[$\star$] Noise is reduced when use higher order accurate
  advection and when move to ALE vertical remapping (with high order
  accurate remapping). 

\item[$\star$] After 20 years of trying, there is no exact and general
  means to locally diagnose
  $\nabla \cdot ( \rho \, C \, {\bm v}^{\dagger})_{\mbox{\tiny
      noise}}$
  in a realistic ocean climate simulation.  However, there are ideas
  that might get us part of the way there (e.g., Ryan Holmes).

\end{itemize}
\end{exampleblock}{}


\end{frame}


\begin{frame}
  \frametitle{Ambiguity with the binning process}

  The discrete water mass transformation expression is given by
\begin{equation}
G(\lambda) 
\approx 
\underbrace{
  \frac{1}{\Delta \lambda} \sum_{n=1}^{N} \sum_{i,j,k}
  \Pi(\lambda_{n},\lambda,\Delta \lambda)
  \, (\rho \, \dot{\lambda})_{i,j,k} \, V_{i,j,k}
}_{\mbox{\tiny Box-car binning of the material change $\rho \dot{\lambda}$  into $N$-bins}}
\label{eq:wmt-discrete-math}
\end{equation}
 with analogous expressions for layer tracer budgets.  

\begin{exampleblock}{}
\begin{itemize}
\item What bin size should we use?  
   \begin{itemize} \footnotesize 
   \item[$\star$] In many cases, bin size can have a nontrivial impact on the results.
   \item[$\star$] Is that because we are using poor numerics?  
   \item[$\star$] Perhaps ALE remapping will help? 
\end{itemize}

\item It is useful for debugging the binning code to check that the
  depth/layer integrated budget adds up to the native model
  budget. 

\item Even so, there is no obvious sanity check on WMT
     within any particular bin.
     \begin{itemize}\footnotesize
     \item[$\star$] We are aiming for quantitatively robust
       results, and yet there is no direct means to double-check
       the numbers. 
     \item[$\star$] Given sensitivities to numerical choices (binning
       algorithm, binning classes), the lack of a double-check is a
       real problem.
     \item[$\star$] The analogous issue does {\it not} arise with
       native-grid budget analyses, since we can check ``left hand
       side = right hand side'' at each grid cell.  
    \end{itemize}

  \item Should we consider a detailed best-practices guide for WMT analysis?

\end{itemize}
\end{exampleblock}{}


\end{frame}



\begin{frame}
  \frametitle{What is the best globally defined buoyancy?}

  For buoyancy-based WMT analysis, we remain using non-optimal density
  fields such as $\gamma^{\mbox{\tiny n}}$ or potential density.

\begin{exampleblock}{}
\begin{itemize}

\item A wish-list for a new-and-improved $\gamma$ field.


\begin{itemize}  \footnotesize 
\item[$\star$] A state-dependent equation for $\gamma$ to allow online
  binning.

\item[$\star$] An expression for $\gamma$ that is general enough to
  allow for arbitrary model state: paleoclimate and climate change.

\item[$\star$] No direct connection to observational data so that
  there is no geographical dependence.

\item[$\star$] Accurate enough so that the material changes in
  $\gamma$ are very close to those of locally referenced potential
  density. 

\item[$\star$] Stated alternatively, we wish to have the $b-$factor
  from \href{https://doi.org/10.1357/0022240053693734}{McDougall and
    Jackett (2005)} and
  \href{https://journals.ametsoc.org/doi/10.1175/2007JPO3464.1}{Iudicone
    et al (2008)} near unity everywhere
\begin{equation}
   \frac{\dot{\gamma}}{\gamma} = b \, \frac{\dot{\rho}^{\mbox{\tiny local}}}{\rho^{\mbox{\tiny local}} }
   = b \, (-\alpha \, \dot{\Theta} + \beta \, \dot{S})  
  \qquad b = \frac{ | \nabla \gamma| }{ | \nabla \rho^{\mbox{\tiny local}} | } 
  \approx 1.
\end{equation}

\end{itemize}

\item Geoff Stanley's ongoing work with Trevor to generalize the
  \href{https://www.ocean-sci.net/5/155/2009/}{Klocker and McDougall
    (2009)} $\omega$-surfaces holds some promise.

\end{itemize}
\end{exampleblock}{}

\end{frame}



\begin{frame}
\frametitle{Many thanks for your time and attention}

\begin{center}
\vspace{-.25cm}
\fbox{\includegraphics[width=1.0\textwidth]{../../../../Griffies/talks_meetings/2019/Australia_AMSI_lecture/talk/images/pictures/Southern_Ocean_photos.pdf}}
\end{center}
\center{\scriptsize From the Weddell Sea and Scotia Sea, autumn 2017 on the RRS James Clark Ross}


\end{frame}  





\end{document}

