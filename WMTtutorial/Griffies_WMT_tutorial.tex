%LATEX 
%\documentclass[10pt,draft]{beamer}
\documentclass[10pt]{beamer}
\usepackage{beamerthemesplit} % new 

\usepackage{bm,amsmath,amssymb}
\usepackage{mathtools,slashed,esint}
\usepackage{animate}

\usepackage{pgf}
\usepackage{wasysym}
\usepackage{xcolor}
\usepackage[english]{babel}
\usepackage[latin1]{inputenc}
\usepackage{times}
\usepackage[T1]{fontenc}
\usepackage{wasysym}
\usepackage{centernot}

\usepackage{tikz}
\usepackage{pgf}
\usepackage{xcolor}
\usepackage[english]{babel}
\usepackage[latin1]{inputenc}
\usepackage{times}
\usepackage[T1]{fontenc}

%\usepackage{graphics}
\usepackage{graphicx}
\usepackage{media9}
\usepackage{natbib}
\usepackage{color}

%\usepackage{../../../../../Documents/TEX/styles/mystyle_beamer}

\usepackage{ragged2e}
\usepackage{etoolbox}
\usepackage{lipsum}

\apptocmd{\frame}{\justifying}{}{}

\usepackage{hyperref}  % must be used last in the list of packages 
\hypersetup{colorlinks=true,linkcolor=cyan,citecolor=cyan,filecolor=cyan,urlcolor=cyan}
% \hypersetup{colorlinks=true,linkcolor=red,citecolor=red,filecolor=red,urlcolor=red} 


\setlength\fboxsep{0pt}
\setlength\fboxrule{0.5pt}

\newcommand\myeq{\mathrel{\overset{\makebox[0pt]{\mbox{\normalfont\tiny\sffamily M constant}}}{=}}}
\newcommand\myeqq{\mathrel{\overset{\makebox[0pt]{\mbox{\normalfont\tiny\sffamily Stokes Thm}}}{=}}}

\newcommand{\notimplies}{%
  \mathrel{{\ooalign{\hidewidth$\not\phantom{=}$\hidewidth\cr$\implies$}}}}

\newcommand{\Simley}[1]{%
\begin{tikzpicture}[scale=0.11]
    \newcommand*{\SmileyRadius}{1.0}%
    \draw [fill=brown!10] (0,0) circle (\SmileyRadius)% outside circle
        %node [yshift=-0.22*\SmileyRadius cm] {\tiny #1}% uncomment this to see the smile factor
        ;  

    \pgfmathsetmacro{\eyeX}{0.5*\SmileyRadius*cos(30)}
    \pgfmathsetmacro{\eyeY}{0.5*\SmileyRadius*sin(30)}
    \draw [fill=cyan,draw=none] (\eyeX,\eyeY) circle (0.15cm);
    \draw [fill=cyan,draw=none] (-\eyeX,\eyeY) circle (0.15cm);

    \pgfmathsetmacro{\xScale}{2*\eyeX/180}
    \pgfmathsetmacro{\yScale}{1.0*\eyeY}
    \draw[color=red, domain=-\eyeX:\eyeX]   
        plot ({\x},{
            -0.1+#1*0.15 % shift the smiley as smile decreases
            -#1*1.75*\yScale*(sin((\x+\eyeX)/\xScale))-\eyeY});
\end{tikzpicture}%
}%



% beamer prologue begins here
\mode<presentation>
{                               % see user guide for choices
%  \usetheme{Frankfurt}
  \usetheme{Madrid}
%  \usecolortheme{seahorse}
%  \usecolortheme[RGB={205,173,0}]{structure} 
  \usecolortheme[RGB={0,100,180}]{structure} 
  \usefonttheme[onlymath]{serif}
  \setbeamercovered{transparent} % or whatever (possibly just delete it)
}

% drop those navigation symbols at bottom right
%\setbeamertemplate{navigation symbols}{} 
\setbeamertemplate{navigation symbols}{}
\setbeamertemplate{footline}[frame number]


\title[WMT Theory and Practice]{A technical guide to water mass transformation (WMT) analysis} % (optional)
\author[{\sc
  Stephen.M.Griffies@gmail.com}]{ {\sc Stephen.Griffies@noaa.gov} 
 \\ \vspace{.3cm} {\small  \today}
}
% - Use the \inst{?} command only if the authors have different 
%   affiliation. 

\institute[NOAA/GFDL + Princeton University]{NOAA/GFDL + Princeton University} 

%\date{\today}
\date{}


%\logo{\includegraphics[height=5mm]{../../../../../Documents/images/PrincetonLogo.png}}  
%\logo{\includegraphics[height=5mm]{./figs/noaa-logo.png}} 

\logo{\includegraphics[height=3mm]{./figs/noaa-logo.png}  \hspace{11cm} 
\includegraphics[height=3mm]{./figs/PrincetonLogo.png}} 

\subject{Talks} % for "PDF information catalog", can be left out. 

% this shows the outline at the beginning of every section,
% highlighting the current section
 \AtBeginSection[]
 {
   \begin{frame}
     \frametitle{Outline}
 \tableofcontents[ 
 currentsubsection, 
 hideothersubsections, 
 sectionstyle=show/hide, 
 subsectionstyle=show/hide, 
 ] 
   \end{frame}
 }


\setlength{\unitlength}{1mm}  % used in picture environment

\begin{document}

%\footnotesize 

  
% beamer makes the titlepage from info above: author, date, title, etc.
\begin{frame}
  \titlepage
\end{frame}



% beamer makes the ToC from sections and subsections below...
% \begin{frame}
%   \frametitle{Outline}
%   \tableofcontents
% \end{frame}


%\section{Fundamentals} 

\begin{frame}
  \frametitle{Some general points}
\scriptsize 

\begin{exampleblock}{}
  
  \begin{itemize}
    
  \item[$\star$] Water mass transformation (WMT) analysis is the study
    of budgets for seawater mass and/or tracer mass within layers or \underline{\it
      water mass classes}.  

   \begin{itemize}
   \scriptsize 
   \item This approach contrasts to local (Eulerian) budget analysis
     that focuses on fixed regions.
   \item It also contrasts with Lagrangian or quasi-Lagrangian
     approaches that follow fluid elements.
   \end{itemize}

 \item[$\star$] Layers or classes are defined by a scalar field
   $\lambda$, which is a function of space and time,
   $\lambda({\bm x},t)$.
    \begin{itemize}
      \scriptsize 
     \item tracer concentrations: $\lambda = S, \Theta, \mbox{CO}_{2}$ 
     \item buoyancy: $\lambda = \gamma$ = potential density or neutral density 
    \end{itemize}
  
  \item[$\star$] WMT analysis is concerned with the \underline{\it
      mass distribution} of seawater binned according to
    $\lambda$-classes.
\begin{itemize}
      \scriptsize 
    \item By extension, WMT analysis is concerned with the physical
      and biogeochemical processes leading to modifications of the
      mass distribution.
    \end{itemize}

  \item[$\star$] \underline{\it Transformation} refers to the
    transport of seawater mass and tracer mass across a
    $\lambda$-isosurface, thus leading to a modification of the
    $\lambda$ mass distribution.
\begin{itemize}
      \scriptsize 
    \item Transformation across a $\lambda$-isosurface is measured by
      a nonzero material change to $\lambda$.
     \item Written mathematically, transformation occurs when 
\begin{equation*}
  \dot{\lambda}  =  \frac{\mathrm{D} \lambda}{ \mathrm{D}t}
  = \frac{\partial \lambda}{\partial t} + {\bm v} \cdot \nabla \lambda \ne 0.
\end{equation*}
    \end{itemize}


  \item[$\star$] \underline{\it Formation} refers to the net
    accumulation of water into a layer or class.
\begin{itemize}
      \scriptsize 
    \item Formation is computed by taking the difference of the
      transformation across the layer boundaries.
     \item \underline{\it Formation = convergence of transformation}.
\end{itemize}


\end{itemize}
\end{exampleblock}{}

\end{frame}


\begin{frame}
  \frametitle{Mass distributions, transformation, and formation}
\scriptsize 

\begin{exampleblock}{}
\begin{itemize}
\item[$\star$]
  $\mathrm{d} M = m(\lambda) \, \delta \lambda = \mbox{mass within the
    infinitesimal $\lambda$-layer} \; [\lambda - \delta \lambda/2,
  \lambda + \delta \lambda/2]$
\item[$\star$]  $G(\lambda) = $ {\it transformation} = mass transport
 (mass per time) of seawater crossing $\lambda$-isosurface.
\item[$\star$]  $G(\lambda) > 0$ (conventionally) means water moves to larger $\lambda$.  
\item[$\star$] We here consider a three-class example with $\lambda = \gamma = $ buoyancy.
\end{itemize}
\end{exampleblock}

\begin{center}
{\includegraphics[width=0.9\textwidth]{./figs/sample_density_distribution.pdf}}
\end{center}

\begin{equation*}
   G(\gamma') = 
    \left[
    \begin{array}{lll}
     0     &\gamma' = \gamma - \delta\gamma/2  &\mbox{boundary assumed to be closed} \\
     <0   &\gamma' = \gamma + \delta\gamma/2  &\mbox{mass moves to light density from middle}\\
     >0   &\gamma' = \gamma + 3 \, \delta\gamma/2  &\mbox{mass moves from middle density to heavy}\\
     0     &\gamma' = \gamma + 5\, \delta\gamma/2  &\mbox{boundary assumed to be closed} 
    \end{array}
    \right.
\end{equation*}

The convergence of $G$ yields the formation into a layer: 
\begin{align*}
  \mbox{light-formation}  &= -[G(\gamma + \delta\gamma/2) - G(\gamma - \delta\gamma/2)] > 0 
\\
  \mbox{middle-formation} &= -[G(\gamma  + 3 \, \delta\gamma/2) - G(\gamma + \delta\gamma/2)]  < 0 
\\
  \mbox{heavy-formation}  &= -[G(\gamma  + 5 \, \delta\gamma/2) - G(\gamma + 3 \, \delta\gamma/2)] > 0.  
\end{align*}


\end{frame}


\begin{frame}
  \frametitle{Seawater mass budget for a $\Delta \lambda$-layer}

\scriptsize 
\begin{exampleblock}{}

\begin{equation*}
 \underbrace{ \frac{\mathrm{d} \Delta M}{\mathrm{d}  t} + \Delta \Psi}_{\mbox{\tiny storage + outflow}}
 = \underbrace{ \Delta W 
 -[ G(\lambda + \Delta \lambda/2) - G(\lambda - \Delta \lambda/2) ]
  }_{\mbox{\tiny formation into layer $\Omega_{(\lambda \pm \Delta \lambda/2)}$}}
\end{equation*}

\tiny 
\end{exampleblock}
\begin{center}
{\includegraphics[width=0.4\textwidth]{./figs/wmt_lambda_schematic.pdf}}
\hspace{.5cm}
{\includegraphics[width=0.5\textwidth]{./figs/tracer_mass_density_function.pdf}}
 \end{center}

\tiny 

\begin{align*}
 \underbrace{\Delta M(\lambda \pm \Delta \lambda/2)}_{\mbox{\tiny layer mass}}
 &= \int_{\Omega(\lambda \pm \Delta \lambda/2)} \rho \, \mathrm{d}V 
   =  \int_{\lambda - \Delta \lambda/2}^{\lambda + \Delta \lambda/2}  m(\lambda') \, \mathrm{d}\lambda'
\\
 \underbrace{\Delta \Psi(\lambda \pm \Delta \lambda/2)}_{\mbox{\tiny interior bdy mass transport}}
 &= +\int_{\partial \Omega_{\mbox{\tiny in}}(\lambda \pm \Delta \lambda/2)} \rho \,
 ({\bm v} - {\bm v}^{(b)}) \cdot \hat{\bm n} \, \mathrm{d}\mathcal{S}
= \int_{\lambda - \Delta \lambda/2}^{\lambda + \Delta \lambda/2} 
    \dot{m}_{\Psi}(\lambda') \, \mathrm{d}\lambda'
\\
 \underbrace{\Delta W(\lambda \pm \Delta \lambda/2)}_{\mbox{\tiny surface bdy mass transport}}
&=
-\int_{\partial \Omega_{\mbox{\tiny out}}(\lambda \pm \Delta \lambda/2)} \rho \,
 ({\bm v} - {\bm v}^{(\eta)}) \cdot \hat{\bm n} \, \mathrm{d}\mathcal{S}
 =
 \int_{\partial \Omega_{\mbox{\tiny out}}(\lambda \pm \Delta \lambda/2)} 
 \mathcal{Q}_{\mbox{\tiny m}} \, \mathrm{d}\mathcal{S}
= \int_{\lambda - \Delta \lambda/2}^{\lambda + \Delta \lambda/2} 
    \dot{m}_{\mbox{\tiny W}}(\lambda') \, \mathrm{d}\lambda'
\\
 m (\lambda) \, \delta \lambda 
&= \mbox{mass contained within the increment $[\lambda-\delta \lambda/2, \lambda+\delta \lambda/2]$}
\\
\dot{m}_{\Psi}(\lambda) \, \delta \lambda 
&= \mbox{mass per time crossing  
$\partial \Omega_{\mbox{\tiny in}}$ 
 within the increment $[\lambda-\delta \lambda/2, \lambda+\delta \lambda/2]$}
\\
 \dot{m}_{\mbox{\tiny W}}(\lambda)   \, \delta \lambda 
&= \mbox{mass per time crossing $\partial \Omega_{\mbox{\tiny out}}$ 
 within the increment $[\lambda-\delta \lambda/2, \lambda+\delta \lambda/2]$.}
\end{align*}   

\end{frame}


\begin{frame}
  \frametitle{Calculating the transformation $G(\lambda)$}
\scriptsize 
 
\begin{exampleblock}{}

Transformation across a $\lambda$-surface is given by the equivalent
mathematical expressions
\begin{equation*}
 G(\lambda) 
 = \int_{\partial \Omega(\lambda) } \rho \,  \hat{\bm n} \cdot ({\bm v} -{\bm v}^{(\lambda)} )   \, \mathrm{d}\mathcal{S}
 = \int_{\partial \Omega(\lambda)} \frac{\rho \, \dot{\lambda}}{|\nabla \lambda|} \, \mathrm{d}\mathcal{S}
 = \lim_{\delta \lambda \rightarrow 0}
 \frac{1}{\delta \lambda}
 \int_{\Omega(\lambda \pm \delta \lambda/2)} \rho \, \dot{\lambda}' \, \mathrm{d}V.
\end{equation*}
The final equality is conducive to estimations via partitioning the
contributions to $\rho \, \dot{\lambda}$ into $\delta \lambda$-bins.

\end{exampleblock}

There are two conceptually distinct but mathematically equivalent
methods to compute $\rho \, \dot{\lambda}$.
\begin{itemize}

\item[$\star$] The \underline{\it process method} tells us why there is
  transformation across a $\lambda$-surface.  This is the method most
  commonly used in practice.
\begin{equation*}
 \rho \, \mathrm{D}\lambda/ \mathrm{D}t 
   = \underbrace{-\nabla \cdot {\bm J}}_{\mbox{\tiny mixing + boundary fluxes}} \; \; 
   + \; \; \underbrace{\rho \, \dot{\Upsilon}}_{\mbox{\tiny sources}}
\end{equation*}

\item[$\star$] The \underline{\it kinematic method} tells us how there is
  transformation across a $\lambda$-surface.
\begin{equation*}
 \rho \, \mathrm{D}\lambda/\mathrm{D}t 
 = \underbrace{\partial (\rho \, \lambda)/\partial t}_{\mbox{\tiny local tendency}}
  \; \; 
  + 
 \; \;  \underbrace{\nabla \cdot (\rho \, {\bm v}^{\dagger} \, \lambda)}_{\mbox{\tiny residual mean advection}} 
\end{equation*}
 
\item[$\star$] Sometimes useful to diagnose the kinematic method and
  then to infer process method mixing.
 
\item[$\star$] Experience with MOM5 suggests that it is not trivial to
  ensure the two methods yield identical results since the diagnostic
  methods must be carefully coded. 

\item[$\star$] We focus on the process method in the remainder of this
  tutorial.

\item[$\star$] The source term,
  $\rho \, \dot{\Upsilon}_{\mbox{\tiny sources}}$, is zero for $S$ and
  $\Theta$, so we ignore it for the remainder.

\end{itemize}


\end{frame}



\begin{frame}
  \frametitle{The process method for computing $G(\lambda)$}
\scriptsize 

\begin{exampleblock}{}
 
  In the absence of interiors sources, the process method consists of
  estimating the following volume integral
\begin{equation*}
 G(\lambda) 
 = 
\lim_{\delta \lambda \rightarrow 0}
 \frac{1}{\delta \lambda}
 \int_{\Omega(\lambda \pm \delta \lambda/2)} \rho \, \dot{\lambda}' \, \mathrm{d}V
=
\lim_{\delta \lambda \rightarrow 0}
 \frac{1}{\delta \lambda}
 \int_{\Omega(\lambda \pm \delta \lambda/2)} [-\nabla \cdot {\bm J} ] \, \mathrm{d}V
\end{equation*}

\end{exampleblock}

It is convenient to decompose the flux divergence into an interior
mixing term plus surface boundary fluxes 
\begin{equation*}
  \nabla \cdot {\bm J} =    \underbrace{ \nabla \cdot {\bm J}_{\mbox{\tiny interior}}
     }_{\mbox{\tiny interior processes}} 
  + \underbrace{ {\bm J}_{\mbox{\tiny out}} \cdot \hat{\bm n} \, \delta (z-\eta)
      }_{\mbox{\tiny surface boundary fluxes}}
  + \underbrace{ {\bm J}_{\mbox{\tiny bot}} \cdot \hat{\bm n} \, \delta (z-\eta_{\mbox{\tiny b}})
     }_{\mbox{\tiny bottom boundary processes}}
\end{equation*}
\begin{multline*}
 G(\lambda)
 =
 \underbrace{
  -\lim_{\delta \lambda \rightarrow 0}
 \frac{1}{\delta \lambda}
 \left[ \int_{\Omega(\lambda \pm \delta \lambda/2)} \nabla \cdot {\bm J}_{\mbox{\tiny interior}} \, \mathrm{d}V
  \right]
  }_{\mbox{\tiny interior transformation = volume integral of convergence}}
 \\
 \underbrace{
  -\lim_{\delta \lambda \rightarrow 0}
 \frac{1}{\delta \lambda}
 \left[ \int_{\partial \Omega_{\mbox{\tiny out}} (\lambda \pm \delta \lambda/2)} 
{\bm J}_{\mbox{\tiny out}} \cdot \hat{\bm n} \, \mathrm{d}\mathcal{S}
  \right]
 }_{\mbox{\tiny surface transformation = area integral of surface fluxes}}
\; \;
 \underbrace{
  -\lim_{\delta \lambda \rightarrow 0}
 \frac{1}{\delta \lambda}
 \left[ \int_{\partial \Omega_{\mbox{\tiny  bot}} (\lambda \pm \delta \lambda/2)} 
{\bm J}_{\mbox{\tiny bot}} \cdot \hat{\bm n} \, \mathrm{d}\mathcal{S}
  \right]
 }_{\mbox{\tiny bottom transformation = area integral of bottom fluxes}}
\end{multline*}

\begin{itemize}
\item[$\star$] The first term on the right hand side is the binned tendency from
interior processes such as mixing and shortwave heating.  

\item[$\star$] {\it Interior} refers to anything not arising from
  fluxes crossing $z=\eta$ or $z-\eta_{\mbox{\tiny b}}$.  Hence, it
  includes strong boundary layer mixing processes.

\item[$\star$] The second term is the binned non-advective surface
transport and third is binned non-advective bottom transport.  

\end{itemize}


\end{frame}


\begin{frame}
  \frametitle{Non-advective surface boundary fluxes}
\tiny 

\begin{exampleblock}{}
 
\begin{equation*}
\mbox{surface transformation}
 = 
 G(\lambda)_{\mbox{\tiny surface}}  
\equiv 
\lim_{\delta \lambda \rightarrow 0}
 \frac{1}{\delta \lambda}
 \left[ \int_{\partial \Omega_{\mbox{\tiny out}} (\lambda \pm \delta \lambda/2)} 
 [-{\bm J}_{\mbox{\tiny out}} \cdot \hat{\bm n}] \, \mathrm{d}\mathcal{S}
  \right]
\end{equation*}

\end{exampleblock}


\begin{itemize}

\item[$\star$] {\sc salinity}: $\lambda = S$.  If salt is dissolved in
  the surface mass flux, then the advective salt flux crossing the
  ocean surface is
  $S_{\mbox{\tiny m}} \, \mathcal{Q}_{\mbox{\tiny m}}$.  There may
  also be a non-advective salt flux,
  $\mathcal{Q}_{\mbox{\tiny S}}^{\mbox{\tiny non-adv}}$, from
  turbulent transfer and/or surface salt restoring in an OMIP
  simulation so that the net salt flux is
\begin{equation*} 
   \mathcal{Q}_{\mbox{\tiny S}} = S_{\mbox{\tiny m}} \,
  \mathcal{Q}_{\mbox{\tiny m}} + \mathcal{Q}_{\mbox{\tiny S}}^{\mbox{\tiny non-adv}}.
\end{equation*}
This salt flux is met on the ocean side by an advective and
non-advective ocean flux so that
\begin{equation*}
  \mathcal{Q}_{\mbox{\tiny S}} = 
   S_{\mbox{\tiny m}} \, \mathcal{Q}_{\mbox{\tiny m}} 
+ \mathcal{Q}_{\mbox{\tiny S}}^{\mbox{\tiny non-adv}} 
= S \, \mathcal{Q}_{\mbox{\tiny  m}} - \hat{\bm n} \cdot {\bm J}^{(S)}
\qquad \mbox{with $S$ the salinity at $z=\eta$.}
\end{equation*}
Rearrangement leads to the induced non-advective flux on the ocean
side of the surface boundary
\begin{equation*}
\boxed{
  -\hat{\bm n} \cdot {\bm J}^{(S)} =
 (S_{\mbox{\tiny m}} - S) \, \mathcal{Q}_{\mbox{\tiny m}} + \mathcal{Q}_{\mbox{\tiny S}}^{\mbox{\tiny non-adv}}.
}
 \end{equation*}
 Typically $S_{\mbox{\tiny m}} =0$ except for exchanges of salt with
 sea ice.

\begin{center}
{\includegraphics[width=0.5\textwidth]{./figs/ocean_surface_boundary_water_Theta_S.pdf}}
 \end{center}


\item[$\star$] {\sc Conservative Temperature}: $\lambda = \Theta$.
  The same considerations hold for Conservative Temperature so that
  its net flux is
\begin{equation*}  
\mathcal{Q}_{\Theta} = \Theta_{\mbox{\tiny m}} \,
  \mathcal{Q}_{\mbox{\tiny m}} + \mathcal{Q}_{\Theta}^{\mbox{\tiny non-adv}}, 
\end{equation*}
where $\mathcal{Q}_{\Theta}^{\mbox{\tiny non-adv}}$ arises from
turbulent and radiative heat fluxes.  This flux is met on the ocean
side by an advective and non-advective ocean flux,
$\mathcal{Q}_{\Theta} = \Theta \, \mathcal{Q}_{\mbox{\tiny m}} -
\hat{\bm n} \cdot {\bm J}^{(\Theta)}$, so that the induced
non-advective flux is
\begin{equation*}
\boxed{
  -\hat{\bm n} \cdot {\bm J}^{(\Theta)} = 
 (\Theta_{\mbox{\tiny m}} - \Theta) \, \mathcal{Q}_{\mbox{\tiny m}} + \mathcal{Q}_{\Theta}^{\mbox{\tiny non-adv}}
 \qquad \mbox{with $\Theta$ the Conservative Temp at $z=\eta$.}
}
\end{equation*}
In most applications we set $\Theta_{\mbox{\tiny m}} - \Theta =0$ so
that
$-\hat{\bm n} \cdot {\bm J}^{(\Theta)} =
\mathcal{Q}_{\Theta}^{\mbox{\tiny non-adv}}$.

\end{itemize}

\end{frame}






\begin{frame}
  \frametitle{Buoyancy water mass transformation: $\lambda = \gamma$}

%\begin{exampleblock}{}

\scriptsize 
$\lambda = \gamma$ is potential density or neutral density, used here to measure buoyancy water mass transformation.   

\tiny  
\begin{align*}
  \rho \, \dot{\gamma} &= 
  \frac{\partial \gamma}{\partial S} \, \rho \, \dot{S}
 +
  \frac{\partial \gamma}{\partial \Theta} \, \rho \, \dot{\Theta}
 =
 \gamma \, ( \beta_{\mbox{\tiny S}} \, \rho \, \dot{S} - \alpha_{\Theta} \, \rho \, \dot{\Theta})
\\
   \rho \, \dot{S} &= 
   - \nabla \cdot {\bm J}^{(S)}_{\mbox{\tiny interior}} 
   - {\bm J}^{(S)}_{\mbox{\tiny out}} \cdot \hat{\bm n} \, \delta(z-\eta)
   - {\bm J}^{(S)}_{\mbox{\tiny bot}} \cdot \hat{\bm n} \, \delta(z-\eta_{\mbox{\tiny b}})
 \\
  \rho \, \dot{\Theta} &= 
    -\nabla \cdot {\bm J}^{(\Theta)}_{\mbox{\tiny interior}} 
    - {\bm J}^{(\Theta)}_{\mbox{\tiny out}} \cdot \hat{\bm n} \, \delta(z-\eta)
   - {\bm J}^{(\Theta)}_{\mbox{\tiny bot}} \cdot \hat{\bm n} \, \delta(z-\eta_{\mbox{\tiny b}} )
\\
% \end{align*}
% %\end{exampleblock}
% \begin{align*}
 G(\gamma) &= 
 \underbrace{
  -\lim_{\delta \gamma \rightarrow 0}
 \frac{1}{\delta \gamma}
 \left[ \int_{\Omega(\gamma \pm \delta \gamma/2)} 
  \left(\frac{\partial \gamma}{\partial S} \, \nabla \cdot {\bm J}^{(S)}_{\mbox{\tiny interior}} 
 +\frac{\partial \gamma}{\partial \Theta} \, \nabla \cdot {\bm J}^{(\Theta)}_{\mbox{\tiny interior}} 
  \right)
  \mathrm{d}V
  \right]
  }_{\mbox{\tiny interior buoyancy transformation = volume integral of convergence}}
\nonumber 
 \\
  & \; \; 
 \underbrace{
  -\lim_{\delta \gamma \rightarrow 0}
 \frac{1}{\delta \gamma}
 \left[ \int_{\partial \Omega_{\mbox{\tiny out}} (\gamma \pm \delta \gamma/2)} 
 \left(  \frac{\partial \gamma}{\partial S} \, {\bm J}^{(S)}_{\mbox{\tiny out}}  
     +   \frac{\partial \gamma}{\partial \Theta} \, {\bm J}^{(\Theta)}_{\mbox{\tiny out}}  
  \right)
  \cdot \hat{\bm n} \, \mathrm{d}\mathcal{S}
  \right]
 }_{\mbox{\tiny surface buoyancy transformation = area integral of surface boundary fluxes}}
\nonumber 
\\
 & \; \; 
  \underbrace{
 -\lim_{\delta \gamma \rightarrow 0}
 \frac{1}{\delta \gamma}
 \left[ \int_{\partial \Omega_{\mbox{\tiny bot}} (\gamma \pm \delta \gamma/2)} 
 \left(  \frac{\partial \gamma}{\partial S} \, {\bm J}^{(S)}_{\mbox{\tiny bot}}  
     +   \frac{\partial \gamma}{\partial \Theta} \, {\bm J}^{(\Theta)}_{\mbox{\tiny bot}}  
  \right)
  \cdot \hat{\bm n} \, \mathrm{d}\mathcal{S}
  \right]
 }_{\mbox{\tiny bottom buoyancy transformation = area integral of bottom boundary fluxes}}
\\
G(\gamma)_{\mbox{\tiny surface}}^{\mbox{\tiny S}} &=
 \lim_{\delta \gamma \rightarrow 0}
 \frac{1}{\delta \gamma}
 \left[ \int_{\partial \Omega_{\mbox{\tiny out}} (\gamma \pm \delta \gamma/2)} 
  \beta_{\mbox{\tiny S}} \, 
          [\mathcal{Q}_{S}^{\mbox{\tiny non-adv}} + (S -  S_{\mbox{\tiny m}} ) \, \mathcal{Q}_{\mbox{\tiny m}}]  
  \gamma \, \mathrm{d}\mathcal{S}
  \right]
\\
G(\gamma)_{\mbox{\tiny surface}}^{\Theta} &=
 -\lim_{\delta \gamma \rightarrow 0}
 \frac{1}{\delta \gamma}
 \left[ \int_{\partial \Omega_{\mbox{\tiny out}} (\gamma \pm \delta \gamma/2)} 
  \alpha_{\Theta} \, 
       [\mathcal{Q}_{\Theta}^{\mbox{\tiny non-adv}} +  (\Theta_{\mbox{\tiny m}}-\Theta) \, \mathcal{Q}_{\mbox{\tiny m}}]
  \gamma \, \mathrm{d}\mathcal{S}
  \right]
\end{align*}

\scriptsize 
\begin{itemize}

\item[$\star$] Note multiplication of terms by
  $(\partial \gamma/\partial S)$ and
  $(\partial \gamma/\partial \Theta)$. Offline now (online requires
  MOM6 code).

\item[$\star$] $\Theta_{\mbox{\tiny m}}-\Theta = 0$ assumed for surface advective heat fluxes. 

\item[$\star$] $S_{\mbox{\tiny m}} = 0$ assumed for precip, evap, and rivers. 

\end{itemize}

\end{frame}


\begin{frame}
  \frametitle{Example surface buoyancy transformation}
 
\begin{figure}
\centering
\includegraphics[angle=0,width=.7\textwidth]{./figs/WMT_schematic.pdf}
\caption{Transformation due to meridional gradient in the surface
  buoyancy loss. Buoyancy loss is denoted by the blue vertical arrows
  indicating the removal of buoyancy from the ocean. The surface
  buoyancy loss causes $\gamma$ interfaces to migrate to the south,
  which in turn causes dianeutral mass flux to move from lighter
  layers to denser layers (black vectors pointed to the north,
  $w^{\mbox{\tiny dia}}$). With a peak in the buoyancy loss at a
  particular latitude, more entrainment is driven into the layer to
  the north of the peak (water converges to the layer
  $\gamma_{3/2} \le \gamma \le \gamma_{5/2}$) and less entrainment
  into the layer to the south (water diverges from the layer
  $\gamma_{1/2} \le \gamma \le \gamma_{3/2}$).}
\label{fig:WMT_schematic}
\end{figure}


\end{frame}

\begin{frame}
  \frametitle{Further study}
 
\begin{itemize}

\item[$\star$] Much of the formalism presented here is taken from the
  review paper:
 \begin{itemize}
 \item
   \href{https://www.annualreviews.org/doi/abs/10.1146/annurev-marine-010318-095421}{The
     water mass transformation framework for ocean physics and
     biogeochemistry, 2019: S. Groeskamp, S.M. Grif\/f\/ies,
     D. Iudicone, R. Marsh, A.J.G. Nurser, and J.D. Zika, {\it Annual
       Review of Marine Science}, {\bf 11}, 21.1-21.35,
     doi:10.1146/annurev-marine-010318-095421.}
  \end{itemize}

\item[$\star$] Further material is provided in Chapter 50 of the
  book
 \begin{itemize}
 \item
   \href{https://stephengriffies.github.io/assets/pdfs/GFM\_lectures.pdf}{Elements
     of Geophysical Fluid Mechanics with Ocean Applications, 2020:
     S.M. Grif\/f\/ies}.
  \end{itemize}

\end{itemize}

\end{frame}




\end{document}

